\chapter{Multilinear Algebra}

\section{Tensor Product}
\subsection{Tensor Product}
The tensor product is a universal object in the category, whose objects are multilinear maps of a fixed set of modules $E_1,\cdots,E_n$. The morphism from $f$ to $g$ is defined by $h$ which makes the following diagram commutative: 
\xymatrix{&E_1\times\cdots\times E_n \ar[ld]|f \ar[rd]|g  \\
F\ar@{-}[r]   &h\ar[r] &G }
\par
\begin{theorem}
A tensor product exists and is uniquely determined up to a unique isomorphism.\
\end{theorem}
\begin{proof}
By abstract nonsense, we know of course a tensorproduct is uniquely determined. \par
Let $M$ be the free module generated by the set of all $n$-tuples $(x_1,\cdots.x_n)$,$(x_i\in E_i)$, i.e. generated by the set $E_1\times\cdots\times E_n$. Let $N$ be the submodule generated by all the elements of the following type:
\[(x_1,\cdots,x_i+x'_i,\cdots,x_n)-(x_1,\cdots,x_i,\cdots,x_n)-(x_1,\cdots,x'_i,\cdots,x_n)\]
\[(x_1,a\cdots,x_i,\cdots,x_n)-a(x_1,\cdots,x_i,\cdots,x_n)\]
for all $x_i\in E_i,x'_i\in E_i,a\in R$. We have the canonical injection \[E_1\times\cdots\times E_n\longrightarrow M\]of our set into the free module generated by it. We compose this map with the canonical map $M\longrightarrow M/N$ on the factor module, to get a map \[\varphi : E_1\times\cdots\times E_n\longrightarrow M/N\]
We contend that $\varphi$ is multilinear and is a tensor product.\par
It is obvious that $\varphi$ is multilinear. Our definition was adjusted to this purpose. Let \[f:E_1\times\cdots\times E_n\longrightarrow G\] be a multilinear map. 
\end{proof}
The module $M/N$ will be denoted by \[E_1\otimes\cdots\otimes E_n\]


\subsection{Kronecker Prodcut: Application to Matrix}
\begin{definition}

\end{definition}

\begin{definition}[vectorization]
$\vec:\mathbb{R}^{m}\otimes\mathbb{R}^{n}\cong\mathbb{R}^{mn}$ by converting a matrix to a vector is an isomorphism.
\end{definition}
\begin{theorem}
Suppose $A:k\times l, X:l\times m, B:m\times n$. Then
\[\vec(AXB)=B^{t}\otimes A\vec(X)\]
\end{theorem}
\begin{proof}
Write $X=(x_1,\cdots,x_l)$, where $x_i$ is a $m\times 1$ vector.
\[
RHS=\begin{pmatrix}
 b_{11}A & \cdots & b_{m1}A\\
 \vdots &  & \vdots\\
  b_{1n}A& \cdots & b_{mn}A
\end{pmatrix}
\begin{pmatrix}
 v_1\\
 \vdots\\
v_l
\end{pmatrix}=\begin{pmatrix}
\sum_{i=1}^{m} b_{i1}Av_i \\
 \vdots \\
\sum_{i=1}^{m} b_{in}Av_i
\end{pmatrix}\\\]
\[LHS=\vec((Av_1,\cdots,Av_l)B)=\vec(\sum_{i=1}^{m} b_{i1}Av_i ,\cdots,\sum_{i=1}^{m} b_{in}Av_i)\]
\end{proof}
\begin{lemma}
$\tr(BCD)=(\vec(B^T))^T(I\otimes C)\vec(D)$
\end{lemma}
\begin{proof}
$\tr(BCD)=(\vec(B^T))^T\vec(CD)=(\vec(B^T))^T(I\otimes C)\vec(D)$
\end{proof}




\begin{theorem}
Let $A\in M_{n\times n}(\mathbb{F})$, $B\in M_{q\times q}(\mathbb{F})$. Then the eigenvalues of $A\otimes I_q+I_n\otimes B$ are $\lambda_i(A)+\lambda_j(B)$, $1\le i\le n, 1\le j\le q$.
\end{theorem}
\begin{proof}
Take the Jordan canonical form.
\end{proof}

\begin{theorem}

\end{theorem}
