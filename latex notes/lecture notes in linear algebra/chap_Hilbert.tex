
\chapter{Operators on Hilbert Space}
Often, topics of Hilbert spaces is not contained in a book of linear algebra. But Hilbert Spaces is so much like the finite-dimensional Euclidean spaces, it is natural to begin our journey of functional analysis after the study of inner product spaces. Many of the definitions and theorems are modification of previously established results. 
\section{Hilbert Spaces}
\subsection{Examples}
\begin{example}
    Let $I$ be any set and $l^2(I)$ denotethe set of all functions 
\end{example}

\begin{example}
    Let $(X,\Omega,\mu)$ be a measure space consisting of a set $X$, a $\sigma$-algebra $\Omega$ of subsets of $X$,
    and a countably additive measure $\mu$ defined on $\Omega$ with values in the non-negative extended real numbers
\end{example}

Recall that an absolutely continuous function on the unit interval $[0,1]$ has a derivative a.e. on $[0,1]$.
\begin{example}
    Let $\mathcal{H}=$ the collection of all absolutely continous functions $f:[0,1]\to \mathbb{F}$ s.t. $f(0)=0$
    and $f'\in L^2(0,1)$. If $\left\langle f,g\right\rangle=\int_0^1 f'(t)g'(t)\mathrm{d}t$ for $f$ and $g$ in $\mathcal{H}$,
    then $\mathcal{H}$ is a Hilbert space.
\end{example}

\section{Operators}

