\documentclass{article}
\usepackage[utf8]{inputenc}
\usepackage[a4paper,total={6in,10in}]{geometry}
\usepackage{amsmath}
\usepackage[all,cmtip]{xy}

\usepackage{amsthm}

\newtheorem{theorem}{Theorem}[section]
\newtheorem{notation}{Notation}
\newtheorem{corollary}{Corollary}[theorem]
\newtheorem{lemma}{Lemma}[section]
\newtheorem{example}{Example}[section]
\newtheorem*{remark}{Remark}

\theoremstyle{definition}
\newtheorem{definition}{Definition}[section]


\usepackage{amsfonts}
\usepackage{amssymb}
\usepackage{hyperref}



\title{Sobolev Spaces}
\author{Kaizhao Liu}
\date{April 2023}

\begin{document}
\maketitle

We establish a proper setting in which to apply ideas of functional analysis to glean information concerning PDE

\tableofcontents

\section{H\"older Spaces}
Assume $U\subset \mathbb{R}^n$ is open and $0<\gamma\le 1$. 
\begin{definition}[H\"older continuous]
    A function is said to be H\"older continuous with exponent $\gamma\in (0,1]$ if $\forall x,y\in U$
    \begin{equation}
        |u(x)-u(y)|\le C|x-y|^\gamma
    \end{equation}
    for some constant $C$.    
\end{definition}

\begin{definition}
    (i) If $u:U\to\mathbb{R}$ is bounded and continuous, we write 
    \[\|u\|_{C(\bar{U})}:=\sup_{x\in U}|u(x)|\] 
    (ii) The $\gamma$th-H\"older seminorm of $u:U\to\mathbb{R}$ is 
    \[[u]_{C^{0,\gamma}(\bar{U})}:=\sup_{x,y\in U,x\ne y} \frac{|u(x)-u(y)|}{|x-y|^\gamma}\] 
    and the $\gamma$th-H\"older norm is 
    \[\|u\|_{C^{0,\gamma}(\bar{U})}:= \|u\|_{C(\bar{U})}+[u]_{C^{0,\gamma}(\bar{U})}\]
\end{definition}

\begin{definition}
    The H\"older space \[C^{k,\gamma}(\bar{U})\] consists of all functions $u\in C^k(\bar{U})$ for which the norm 
    \[ \|u\|_{C^{k,\gamma}(\bar{U})}:= \sum_{|\alpha|\le k}\|D^\alpha u\|_{C(\bar{U})}+\sum_{|\alpha|= k}[u]_{C^{0,\gamma}(\bar{U})}\]
    is finite,
\end{definition}

\begin{theorem}[H\"older spaces as function spaces]
    $C^{k,\gamma}(\bar{U})$ is a Banach space.
\end{theorem}
\begin{proof}
    The construction of $\|\cdot\|_{C^{k,\gamma}(\bar{U})}$ ensures that it is a norm. 
    In addition, each Cauchy sequence converges.
\end{proof}



\section{Sobolev Spaces}
The H\"older spaces are unfortunately not often suitable settings for elementary PDE theory,
as we usually cannot make good enough analytic estimates to demonstrate that the solutions we construct actually belong to such spaces.
What are needed rather are some kind of spaces containing less smooth functions.

\subsection{Weak Derivatives}
We start off by substantially weakening the notion of partial derivatives.
\begin{notation}
Let $C_c^\infty(U)$ denote the space of infinitely differentiable function $\varphi:U\to\mathbb{R}$, with compact support in $U$.
We will sometimes call a function belonging to $C_c^\infty(U)$ a test function.
\end{notation}

\begin{definition}
    Suppose 
\end{definition}

\begin{lemma}[uniqueness of weak derivatives]
    A weak $\alpha$th-partial derivative of $u$, if it exists, is uniquely defined up to a set of measure zero.
\end{lemma}

\begin{example}
    Let $n=1$, $U=(0,2)$, and 
    \[ u(x)=\left\{\begin{matrix}
        x  & x\in(0,1]\\
        1  & x\in[1,2)
        \end{matrix}\right.\] 
    has weak derivative 
        \[
            v(x)=\left\{\begin{matrix}
            x  & x\in(0,1]\\
            1  & x\in[1,2)
            \end{matrix}\right.\] 
    But \[ u(x)=\left\{\begin{matrix}
        x  & x\in(0,1]\\
        2  & x\in(1,2)
        \end{matrix}\right.\] 
    does not have a weak derivative.
        
\end{example}

\subsection{Definition of Sobolev Spaces}
Fix $1\le p\le \infty$ and let $k$ be a nonnegative integer. 
\begin{definition}
    The Sobolev space \[W^{k,p}(U)\] consists of all locally summable function $u:U\to\mathbb{R}$ s.t. for each multiindex $\alpha$ with $|\alpha|\le k$,
    $D^\alpha u$ exists in the weak sense and belongs to $L^p(U)$.
\end{definition}
\begin{notation}
    If $p=2$, we usually write \[H^k(U)=W^{k,2}(U)\] 
    $H$ is used since $H^k(U)$ is a Hilbert space.
\end{notation}

\begin{definition}
    If $u\in W^{k,p}(U)$, we define the norm to be 
    \begin{equation}
        \left\{\begin{matrix}
            (\sum_{|\alpha|\le k}\int_U |D^\alpha u|^p\mathrm{d}x)^{\frac{1}{p}}  & p\in [1,\infty)\\
             \sum_{|\alpha|\le k} \text{ess}\sup_U |D^\alpha u| & p=\infty
            \end{matrix}\right.            
    \end{equation}
\end{definition}

\begin{notation}
    
\end{notation}

\begin{notation}
    We denote by \[W_0^{k,p}(U)\] the closure of $C_c^\infty(U)$ in $W^{k,p}(U)$.
\end{notation}
We interpret $W_0^{k,p}(U)$ as comprising those functions $u\in W^{k,p}(U)$ s.t. 

\begin{example}
    If $n=1$ and $U$ is an open interval in $\mathbb{R}^1$
\end{example}

\subsection{Properties}
\begin{theorem}[weak derivatives]
    Assume $u,v\in W^{k,p}(U)$, $|\alpha|\le k$. Then \newline 
    (i) $D^\alpha u\in W^{k-|\alpha|,p}(U)$, and $D^\alpha(D^\beta u) =D^{\alpha+\beta}u$ $\forall |\alpha|+|\beta|\le k$ .\newline 
    (ii) For each $\lambda,\mu\in\mathbb{R}$, $\lambda u+\mu v\in W^{k,p}(U)$ and $D^\alpha(\lambda u+\mu v)=\lambda D^\alpha u+\mu D^\alpha v$, $|\alpha|\le k$.\newline 
    (iii) If $V$ is an open subset of $U$, then $u\in W^{k,p}(V)$.\newline 
    (iv) If $\zeta\in C^\infty_c(U)$, then $\zeta u\in W^{k,p}(U)$ and 
    \[ D^\alpha(\zeta u)=\sum_{\beta\le \alpha}C_\alpha^\beta D^\beta\zeta D^{\alpha-\beta}u \]
\end{theorem}

\begin{theorem}[Sobolev space is Banach]
    For each $k\in\mathbb{Z}_{\ge 1}$ and $1\le p\le\infty$, the Sobolev space $W^{k,p}(U)$ is a Banach space.
\end{theorem}
\begin{proof}
    Note that the completeness is encoded in the definition. If $\{u_m\}_{m=1}^\infty$ is a Cauchy sequence, then for $|\alpha|\le k$, $\{D^\alpha u_m\}_{m=1}^\infty$
    is a Cauchy sequence in $L^p(U)$ and $L^p(U)$ is complete.
\end{proof}

\section{Approximation}
We need to develop some systematic procedures for approximating a function in a Sobolev space by smooth functions. 
The method of mollifiers provides such a tool.
\begin{notation}
    If $U\subset \mathbb{R}^n$ is open and $\epsilon>0$, we write 
    \[ U_\epsilon:=\{x\in U|\text{dist}(x,\partial U)>\epsilon\} \]
\end{notation}

\begin{definition}
    Define $\eta\in C^\infty(\mathbb{R}^n)$ by \[\eta (x)=\left\{\begin{matrix}
        Ce^{\frac{1}{|x|^2-1}} & |x|<1\\
        0 & |x|\ge 1
       \end{matrix}\right.\] 
       $C$ is selected s.t. $\int \eta =1$. We call $\eta$ the standard mollifier.\newline 
       For each $\epsilon>0$, set \[\eta_\epsilon(x)=\frac{1}{\epsilon^n}\eta(\frac{x}{\epsilon})\]
\end{definition}

\begin{definition}
    If $U\to\mathbb{R}$ is locally integrable, define its mollification 
    \[f^\epsilon=\eta_\epsilon*f\quad \text{in} U_\epsilon\]
\end{definition}

\begin{theorem}[properties of mollifiers]
    (i) $f^\epsilon\in C^\infty (U_\epsilon)$\newline 
    (ii) $f^\epsilon\to f$ a.e. as $\epsilon\to 0$\newline 
    (iii) If $f\in C(U)$, then $f^\epsilon\to f$ uniformly on compact subsets of $U$.\newline 
    (iv) If $1\le p<\infty$ and $f\in L^p_{\text{loc}}(U)$, then $f^\epsilon\to f$ in $L^p_{\text{loc}}(U)$.
\end{theorem}

\section{Extensions}
Our goal is to extend functions in the Sobolev space $W^{1,p}(U)$ to become functions in the Sobolev space $W^{1,p}(\mathbb{R}^n)$.
We must invent a way to extend $u$ which preserves the weak derivatives accross $\partial U$.

\begin{theorem}[extension theorem]
    Assume $U$ is bounded and $\partial U$ is $C^1$. Select a bounded open set $V$ s.t. $U\subset \subset V$. Then there exists a bounded linear operator 
    \[ E:W^{1,p}(U)\to W^{1,p}(\mathbb{R}^n)\] 
    s.t. for each $u\in W^{1,p}(U)$:\newline 
    (i) $Eu=u$ a.e. in $U$\newline 
    (ii) $Eu$ has support within $V$\newline 
    (iii) $\|Eu\|_{W^{1,p}(\mathbb{R}^n)}\le C\|u\|_{W^{1,p}(U)}$, where the constant $C$ depends on $p,U,V$.
\end{theorem}


\section{Traces}

\begin{theorem}[trace theorem]
    Assume $U$ is bounded and $\partial U$ is $C^1$. Then there exists a bounded linear operator 
    \[T:W^{1,p}(U)\to L^p(\partial U)\] 
    s.t. \newline 
    (i) $Tu=u|_{\partial U}$ if $u\in W^{1,p}(U)\cap C(\bar{U})$ \newline 
    (ii) $\|Tu\|_{L^p(\partial U)}\le C\|u\|_{W^{1,p}(U)}$
\end{theorem}
\begin{definition}
    We will call $Tu$ the trace of $u$ on $\partial U$.
\end{definition}

\section{Compactness}




\section{Sobolev Inequalities}
Our goal in this section is to discover embeddings of various Sobolev spaces into others. 
The crucical analytic tools here will be so-called Sobolev-type Inequalities, which we prove below for smooth functions.
These will then establish the estimates for arbitary functions in the various relevant Sobolev spaces, since smooth functions are dense.

We will consider first only the Sobolev space $W^{1,p}(U)$ and ask the following 
\subsection{$1\le p<n$}

Let us first ask whether we can establish an estimate of the form 
\[ \|u\|_{L^q(\mathbb{R}^n)}\le C\|Du\|_{L^p(\mathbb{R}^n)}\] 
for certain constants $C>0$, $1\le q<\infty$ and all functions $u\in C^\infty_c(\mathbb{R}^n)$.
By a simple observation, we 
\begin{definition}
    If $1\le p<n$, the Sobolev conjugate of $p$ is 
    \[p^*=\frac{np}{n-p}\] 
\end{definition}

\begin{theorem}[Gagliardo-Nirenberg-Sobolev inequality]
    
\end{theorem}

\begin{theorem}[estimates for $W^{1,p}$]
    Let $U$ be a bounded, open subset of $\mathbb{R}^n$, and suppose $\partial U$ is $C^1$. Assume $1\le p<n$, and $u\in W^{1,p}(U)$.
    Then $u\in L^{p^*}(U)$, with the estimate 
    \[ \|u\|_{L^{p^*}(U)}\le C\|u\|_{W^{1,p}(U)}\] 
    the constant $C$ depending only on $p,n$, and $U$.
\end{theorem}


The Gagliardo-Nirenberg-Sobolev inequality implies the embedding of $W^{1,p}(U)$ into $L^{p^*}(U)$ for $1\le p<n$.
We now demonstrate that $W^{1,p}(U)$ is in fact \textbf{compactly} embedded in $L^q(U)$ for $1\le q<p^*$.
This compactness will be fundamental for our applications of functional analysis to PDE.
\begin{definition}
    Let $X$ and $Y$ be Banach spaces, $X\subset Y$. We say that $X$ is compactly embedded in $Y$, written 
    \[X\subset \subset Y\] 
    provided \newline 
    (i) $\|u\|_Y\le C\|u\|_X$ $(u\in X)$ for some constant $C$ and \newline 
    (ii) each bounded sequence in $X$ is precompact in $Y$.
\end{definition}


\begin{theorem}[Rellich-Kondrachov Compactness Theorem]
    
\end{theorem}


\subsection{$n<p\le\infty$}



\subsection{$p=n$}
Owing to the Gagliardo-Nirenberg-Sobolev inequality, and the fact that $p^*\to\infty$ as $p\to n$, we might expect $u\in L^\infty(U)$ provided 
$u\in W^{1,n}(U)$. This is however false if $n>1$.
\begin{example}
    If $U=B^0(0,1)$, the function $u=\log\log (1+\frac{1}{|x|})$ belongs to $W^{1,n}(U)$ but not to $L^\infty(U)$.
\end{example}

\section{Fourier Transform}
We employ the Fourier transform to give an alternate characterization of the spaces $H^k(\mathbb{R}^n)$.
For this section all functions are complex-valued.
\begin{theorem}[characterization of $H^k$ via Fourier transform]
    Let $k$ be a nonnegative integer.\newline 
    (i) A function $u\in L^2(\mathbb{R}^n)$ belongs to $H^k(\mathbb{R}^n)$ if and only if 
    \[ (1+|y|^k)\hat{u}\in L^2(\mathbb{R}^n)\] 
    (ii) In addition, there exists a positive constant $C$ s.t. 
    \[ \frac{1}{C}\|u\|_{H^k(\mathbb{R}^n)}\le \| (1+|y|^k)\hat{u}\|_{L^2(\mathbb{R}^n)}\le C\|u\|_{H^k(\mathbb{R}^n)} \]
\end{theorem}


\section{The Space $H^{-1}$}
It it important to have an explicit characterization of the dual space of $H^1_0$

\end{document}