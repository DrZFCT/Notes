\chapter{Basic Topology}




To faciliate the study of convergence on an aribitary space

We use metric space as example, for the sake of functional analysis.

\section{Basic Topology}

\subsection{Definitions of Topology}
neighborhood, open set, closed set, 

\subsubsection{Neighborhood}
Let $X$ be a (possibly empty) set. 
Let $\cN$ be a function assigning to each $x\in X$ a non-empty collection $\cN(x)$ of subsets of $X$.
The elements of $\cN(x)$ will be called neighbourhoods of $x$.

\subsubsection{Interior}


\subsubsection{Open}



\subsubsection{Closed}


\subsection{Comparing Different Topologies}

\begin{definition}
    Let $\cT_1$ and $\cT_2$ be two topologies on $X$. We say $\cT_1$ is weaker than $\cT_2$ if $\cT_1\subset \cT_2$.
\end{definition}

\subsection{Constructing New }

\subsubsection{Subspace Topology}

\subsubsection{Product Topology}

\subsection{Topological Basis}


\subsection{Sets in Topological Space}


\section{Compactness}



\begin{definition}[Compact]
    Every open covering has finite sub covering.
\end{definition}
Leveraging the duality of open and closed set, we can immediately

\begin{definition}[Sequentially Compact]
    Every sequence has a convergent subsequence.
\end{definition}

What about subsets? The canonical definition is the following
\begin{remark}
    Let $A\subset X$ be a subset. If $(A,\cT_{subspace})$ is compact, then we say $A$ is compact in $X$.
\end{remark}

We can see that the above definition reduce to the common 
\begin{proposition}
    Let $A\subset X$ be a subset. $A$ is compact in $X$ if and only if 
\end{proposition}


\subsection{Constructing New}

\subsubsection{Subspace of Compact}

\subsubsection{Product of Compact}

\begin{theorem}[Tychonoff]
    If $X_\alpha$ is compact for all $\alpha$, then $(\prod_{\alpha}X_{\alpha},\cT_{product})$ is also compact.
\end{theorem}

\subsection{Compactness in Metric Space}

\begin{theorem}
    In a metric space $(X,d)$, TFAE:
    \begin{enumerate}[label=(\roman*)]
        \item $A$ is compact
        \item $A$ is sequentially compact
        \item $A$ is \textbf{totally} bounded and complete.
    \end{enumerate}
\end{theorem}
\begin{remark}
    Generalization of compactness on $\RR$.
\end{remark}


\section{Separability}

\begin{definition}[first countability]
    
\end{definition}

\begin{definition}[second countability]
    
\end{definition}

