\chapter{Basic Topology}




To faciliate the study of convergence on an aribitary space

We use metric space as example, for the sake of functional analysis.

\section{Basic Topology}

\subsection{Definitions of Topology}
neighborhood, open set, closed set, 

\subsubsection{Neighborhood}
Let $X$ be a (possibly empty) set. 
Let $\cN$ be a function assigning to each $x\in X$ a non-empty collection $\cN(x)$ of subsets of $X$.
The elements of $\cN(x)$ will be called neighbourhoods of $x$.

\begin{enumerate}[label=(\roman*)]
    \item If $N\in \cN(x)$, then $x\in N$.
    \item If $N\subset M$ and $N\in \cN(x)$, then $M\in \cN(x)$.
    \item If $N_1,N_2\in \cN(x)$, then $N_1\cap N_2\in \cN(x)$.
    \item If $N\in \cN(x)$, then there exists $M\in\cN(x)$ such that $M\subset N$, and for any $y\in M$ we have $N\in\cN(y)$.
\end{enumerate}

\subsubsection{Interior}
Given a neighborhood structure, we can define the interior
$$\operatorname{Int}(A):=\{x\in A|A\in\cN(x)\}$$


\begin{enumerate}[label=(\roman*)]
    \item $\operatorname{Int}(A)\subset A$.
    \item $\operatorname{Int}(A)\cap \operatorname{Int}(B)=\operatorname{Int}(A\cap B)$.
    \item $\operatorname{Int}(\operatorname{Int}(A))=\operatorname{Int}(A)$.
    \item $\operatorname{Int}(X)=X$.
\end{enumerate}

Conversely, given an interior structure, we can define a neighborhood structure by
$$\cN(x)=\{A\subset X|x\in\operatorname{Int}{A}\}.$$


\subsubsection{Open Sets}

Given a neighborhood structure, we can define open sets. 
\begin{definition}[Open]
    A set $U$ is \textbf{open} if for any $x\in U$, $U\in\cN(x)$.
\end{definition}
By the equivalence of neighborhood and interior structure,
\begin{proposition}
    A set $U$ is open if and only if $\operatorname{Int}(U)=U$.
\end{proposition}
Denote $$\cT=\{U\subset X|U \text{ is open}\}.$$

\begin{enumerate}[label=(\roman*)]
    \item $\emptyset\in\cT,X\in\cT$.
    \item If $U_1,U_2\in\cT$, then $U_1\cap U_2\in\cT$.
    \item If $\{U_\lambda,\lambda\in\Lambda\}\subset\cT$, then $\cup_{\lambda\in\Lambda}U_\lambda\in\cT$.
\end{enumerate}

Given a topological structure, we cana define a neighborhood
\begin{definition}
    Let $(X,\cT)$ be a topological space, $x\in X$, and $N\subset X$.
    If there exists $U\in\cT$ such that $x\in U\subset N$, then $N$ is a neighborhood of $x$.
\end{definition}
The neighborhood structure can be defined by $$\cN(x)=\{N\subset X|\exists U\in\cT\text{ such that }x\in U \subset N\}.$$

\subsubsection{Closed Sets}
\begin{definition}[Closed]
    Let $F$ be a set in a topological space $(X,\cT)$.
    It is \textbf{closed} if $F^c$ is open.
\end{definition}

\subsection{Comparing Different Topologies}

\begin{definition}
    Let $\cT_1$ and $\cT_2$ be two topologies on $X$. We say $\cT_1$ is weaker than $\cT_2$ if $\cT_1\subset \cT_2$.
\end{definition}

\subsection{Convergence and Continuity}
TODO: add examples

The notion of convergence can be defined when given a topology only.
\begin{definition}[Convergence]
    Let $x_n$ be a sequence in a topological space $X$. 
    If there exists $x_0\in X$ such that for any neighborhood $A$ of $x_0$ there exists $k$ such that $x_n\in A$ when $n>k$, then the sequence $x_n$ \textbf{converges} to $x_0$, denoted by $x_n\to x_0$.
\end{definition}

\begin{definition}[Continuity and Sequential Continuity]
    Let $f:(X,\cT_X)\to(Y,\cT_Y)$
    \begin{enumerate}
        \item continuity
        \item sequential continuity
    \end{enumerate}
\end{definition}
\begin{theorem}
    continuity implies sequential continuity
\end{theorem}


\begin{definition}[Open and Closed Mappings]
    Let $f:(X,\cT_X)\to(Y,\cT_Y)$
    \begin{enumerate}[label=(\roman*)]
        \item If the image $f(U)$ of any open set $U$ is open in $Y$, then $f$ is called an \textbf{open mapping}.
        \item If the image $f(F)$ of any closed set $F$ is closed in $Y$, then $f$ is called an \textbf{closed mapping}.
    \end{enumerate}
\end{definition}


\subsection{Topological Basis}



\begin{equation*}
    \cT_\cB=\{U\subset X|\forall x\in U,\exists B\in\cB \text{ such that } x\in B\subset U\}
\end{equation*}

for which $\cB$, $\cT_\cB$ defined as above is a topology.
\begin{equation*}
    \forall x\in X,\exists B\in\cB\text{ such that }x\in B.
\end{equation*}
\begin{equation*}
    \forall B_1,B_2\in\cB,\forall x\in B_1\cap B_2,\exists B\in\cB\text{ such that }x\in B\subset B_1\cap B_2
\end{equation*}


\begin{theorem}
    \begin{equation*}
        \cT_\cB=\{\cup_{B\in\cB'}B|\cB'\subset \cB\}.
    \end{equation*}
\end{theorem}


\subsection{Constructing New }

\begin{definition}[Subspace Topology]
    Let $(X,\cT)$ be a topological space and $Y\subset X$.
    The set $$\cT_Y:=\{U\cap Y|U\in\cT\}$$ is a toplogy on $Y$, called the \textbf{subspace} topology.
\end{definition}

\begin{definition}[Product Topology]
    Let $(X,\cT_X),(Y,\cT_Y)$ be topological spaces. The set
    $$\cT_{X\times Y}:=\{W\subset X\times Y|\forall (x,y)\in W,\exists U\in\cT_X\text{ and }V\in\cT_Y\text{ such that }(x,y)\in U\times V\subset W\}$$
    is a topology on $X\times Y$, called the \textbf{product toplogy}.
\end{definition}

\begin{definition}
    Let $(Y_\alpha,\cT_\alpha)$ be a family of,
    let 
    $$
    \cF={f_\alpha:X\to (Y_\alpha,\cT_\alpha)}
    $$
    be a family of mappings.
    The the weakest topology on $X$ such that any $f_\alpha$ is continuous
\end{definition}

\begin{example}[Weak and Weak* Topology]
    Let $X$ be a topological vector space and $X^*$ be its dual,
    $$X^*=\{f:X\to\KK|f\text{ linear and continuous}\}.$$
    \begin{enumerate}[label=(\roman*)]
        \item weak topology: $X^*$-induced topology
        \item weak* topology: $\{\ev_x|x\in X\}$-induced topology
    \end{enumerate}
\end{example}



\subsection{Example: Topologies on Mappings}
\label{subsec: toplogies on mappings 1}
For any set $X$ and topological space $Y$, we can consider 
$$
\cM(X,Y):=\{f:X\to Y\}.
$$
We would like to study the topologies on $\cM(X,Y)$ that is associated with the mapping structure.
As $\cM(X,Y)=Y^X$, by the product stucture we can define two topologies on $\cM(X,Y)$:
\begin{enumerate}[label=(\roman*)]
    \item product topology, in this case also the \textbf{pointwise convergence topology} $\cT_{p.c.}$;
    \item box topology, which is not as useful when studying continuous maps.
\end{enumerate}


Now suppose $Y$ is a metric space, then the metric $d_Y$ on $Y$ induce a metric on $\cM(X,Y)$:
$$d_u(f,g):=\sup_{x\in X}\frac{d_Y(f(x),g(x))}{1+d_Y(f(x),g(x))}.$$
We can see that $f_n$ converges to $f$ in this metric if and only if $f_n$ converges uniformly to $f$.
Therefore,
\begin{definition}[Uniform Convergence Topology]
    $d_u$ is called 
\end{definition}

Consider the space of continous mappings
$$
\mathcal{C}(X,Y):=\{f\in \cM(X,Y)|f\text{ continuous}\}.
$$
In general, $\mathcal{C}$ is not closed in $(\cM(X,Y),\cT_{p.c.})$.
\begin{example}
    
\end{example}
But we have 
\begin{proposition}
    $\mathcal{C}$ is closed in $(\cM(X,Y),d_u)$.
\end{proposition}

On $\mathcal{C}(X,Y)$, all of the above topologies are not satisfying:
\begin{example}
    Let $X=Y=\RR$, then 
    \begin{enumerate}
        \item Consider $\cT_{p.c.}$: the sequence of continuous functions $f_n(x)=e^{-nx^2}$ converges to $f_0(x)=\left\{\begin{matrix}
            1, & x=0\\
            0, & x\neq 0
           \end{matrix}\right.,$ which is not continuous. This shows that $\cT_{p.c.}$ is too weak to ensure the continuity of the limit.
        \item Consider $\cT_{u.c.}$ or $\cT_{box}$: the sequence of continuous functions $f_n(x)=x^2/n$ does not converge, where it converges to a 
        continuous function $f_0(x)=0$ under $\cT_{p.c.}$. This shows that these topologies are too strong for sequences to converge.
    \end{enumerate}
\end{example}

What is a good topology on $\mathcal{C}(X,Y)$? We will answer this question in Section \ref{subsec: toplogies on mappings 2}.

compact convergence $\cT_{c.c.}$

when $Y$ is not a metric space
compact open


\subsection{Sets in Topological Space}





\begin{definition}[Limit and Sequential Limit]
    Let $X$ be a topological space, $A\subset X$, and $x\in X$.
    \begin{enumerate}[label=(\roman*)]
        \item If for any neighborhood $U$ of $x$, $$U\cap(A\setminus\{x\})\neq\emptyset,$$ then $x$ is a \textbf{limit point (accumulation point)} of $A$.
        The set of all limit points of $A$ is called the \textbf{derived set} of $A$, denoted by $A'$.
        \item If there exists a sequence $a_n\in A$ such that $a_n\to x$, then $x$ is a \textbf{sequential limit point} of $A$. 
    \end{enumerate}
\end{definition}


A closed set contains all its sequential limit points.
\begin{proposition}
    Let $A\subset X$ be closed. If $x_n\in A$ and $x_n\to x\in X$, then $x\in A$.
\end{proposition}
\begin{proof}
    We prove by contradiction. Suppose $x\in A^c$, as $A^c$ is open, we can find an open neighborhood $U$ of $x_0$ such that $U\subset F^c$.
    By the definition of convergence, there exists $N$ such that for any $n>N$, $x_n\in U$. This is contradictory to $x_n\in F$.
\end{proof}
But the converse is not true in general.
\begin{example}
    Consider the space $X=\cM([0,1],\RR)$ with the pointwise convergence topology.
    \kz{to do 1.119}
\end{example}

A closed set also contains all its limit point. More precisely, we have:
\begin{proposition}
    $A\subset X$ is closed if and only if $A'\subset A$.
\end{proposition}
\begin{proof}
    If $A$ is closed, for any $x\in A^c$ there exists an open neighborhood $U$ of $x$ such that $U\subset A^c$.
    This means that $U\cap A=\emptyset$, so $x\notin A'$ and $A'\subset A$.
    
    If $A'\cap A$, for any $x\in A^c$ there exists a neighborhood $U$ of $x$ such that $U\cap (A\setminus\{x\})=\emptyset$.
    So $U\subset A^c$ and $A$ is closed.
\end{proof}


closure $\overline{A}:=A\cup A'$



\begin{theorem}
    For any $A\subset X$, $A\cup A'$ is closed.
\end{theorem}
\begin{corollary}
    \begin{equation*}
        A\cup A'=\bigcup_{A\subset F \text{ closed }}F.
    \end{equation*}
\end{corollary}

The following proposition provides a way to verify if a point is in the closure of a set.
\begin{proposition}\label{prop:x in overline A}
    $x\in \overline{A}\Longleftrightarrow$ for any open set $U$ contains $x$, $U\cap A\neq\emptyset$.
\end{proposition}
\begin{proof}
    
\end{proof}

We can use the closure to characterize continuous mappings.
\begin{proposition}
    
\end{proposition}

interior is the dual of closure 

\begin{definition}[Dense and Nowhere Dense]
    Let $X$ be a topological space and $A$ be a subset.
    \begin{enumerate}[label=(\roman*)]
        \item If $\overline{A}=X$, then $A$ is called \textbf{dense}.
        \item If $\mathring{\overline{A}}=\emptyset$, then $A$ is called \textbf{meager(nowhere dense)}.
        \item If $A$ is a countable union of meager sets, then $A$ is called a \textbf{Baire first category} set.
    \end{enumerate}
\end{definition}

\begin{definition}
    
\end{definition}


\section{Compactness}



\begin{definition}[Compactness and Sequential Compactness]
    Let $(X,\cT)$ be a topological space.
    \begin{enumerate}[label=(\roman*)]
        \item If every open covering of $X$ has a finite sub-covering, then $X$ is \textbf{compact}.
        \item If every sequence in $X$ has a convergent subsequence, then $X$ is \textbf{sequentially compact}.
    \end{enumerate}
    
\end{definition}
Leveraging the duality of open and closed set, we can immediately


What about subsets? The canonical definition is the following
\begin{remark}
    Let $A\subset X$ be a subset. If $(A,\cT_{subspace})$ is compact, then we say $A$ is compact in $X$.
\end{remark}
We can see that the above definition reduce to the common 
\begin{proposition}
    Let $A\subset X$ be a subset. $A$ is compact in $X$ if and only if 
\end{proposition}
For sequential compactness, the definition is straightforward.


Compactness is preserved under continuous mappings.
\begin{theorem}
    Let $f:X\to Y$ be a continuous map.
    \begin{enumerate}[label=(\roman*)]
        \item If $A\subset X$ is compact, then $f(A)\subset Y$ is compact.
        \item If $A\subset X$ is sequentially compact, then $f(A)\subset Y$ is sequentially compact.
    \end{enumerate}
\end{theorem}
\begin{corollary}
    quotient space
\end{corollary}

Compactness is inherited for closed subsets.
\begin{theorem}
    Let $A\subset X$ be closed.
    \begin{enumerate}[label=(\roman*)]
        \item If $X$ is compact, then $A$ is compact.
        \item If $X$ is sequentially compact, then $A$ is sequentially compact.
    \end{enumerate}
\end{theorem}


\subsection{Product of Compact Spaces: Tychonoff Theorem}

\begin{theorem}[Tychonoff]
    If $X_\alpha$ is compact for all $\alpha$, then $(\prod_{\alpha}X_{\alpha},\cT_{product})$ is also compact.
\end{theorem}

\subsection{Example: Metric Spaces}

Let $(X,d)$ be a metric space.
Compared with a general topological space, a metric space has the following nice properties:
\begin{enumerate}[label=(\roman*)]
    \item \textbf{first countable}, because. Thus,
    \begin{itemize}
        \item $F\subset X$ is closed if and only if it contains all sequential limit points.
        \item For any topological space $Y$, $f:X\to Y$ is continuous if and only if $f$ is sequentially continuous.
    \end{itemize} 
    \item \textbf{Hausdorff}. Thus,
    \begin{itemize}
        \item Compact sets are closed.
        \item A convergent sequence has a unique limit.
    \end{itemize}
\end{enumerate}

\begin{definition}[Total Boundedness]
    Let $(X,d)$ be a metric space. If for any $\epsilon>0$, there exists a finite number of $\epsilon$ balls that cover $X$.
\end{definition}

\begin{definition}[$\epsilon$ net]
    
\end{definition}

Total boundedness is equivalent to finite $\epsilon$ net
\begin{proposition}
    A metric space $X$ is totally bounded if and only if for any $\epsilon>0$, there is a finite $\epsilon$-net in $X$.
\end{proposition}

A topological view of completeness.
\begin{definition}[Absolutely Closed]
    
\end{definition}
\begin{proposition}
    A metric space is absolutely closed if and only if it is complete.
\end{proposition}
\begin{proof}
    Suppose $(X,d)$ is complete, and $(X,d)$ can be ? into $(Y,d_Y)$. Then $X$ is closed in $(Y,d_Y)$ because it contains all 
\end{proof}

\begin{theorem}
    In a metric space $(X,d)$, TFAE:
    \begin{enumerate}[label=(\roman*)]
        \item $A$ is compact.
        \item $A$ is sequentially compact.
        \item $A$ is \textbf{totally} bounded and \textbf{absolutely} closed.
    \end{enumerate}
\end{theorem}
\begin{remark}
    This theorem is a natural generalization of the different characterizations of compactness on $\RR$, which we have learned in mathematical analysis.
\end{remark}

\subsection{Example: Topologies on Mappings}
\label{subsec: toplogies on mappings 2}
This subsection continues the discussion of Section \ref{subsec: toplogies on mappings 1}.
We can try to find a topology on $\cM(X,Y)$ that describes ``uniform convergence on every compact set''.
We can find it by comparing with the construction of $\cT_{p.c.}$ and $\cT_{u.c.}$:
\begin{itemize}
    \item The topological basis of $\cT_{u.c.}$ are the balls $$B(f;X;\epsilon)=\{g\in\cM(X,Y)|\sup_{x\in X}d(f(x),g(x))<\epsilon\}.$$
    \item The topological basis of $\cT_{p.c.}$, i.e. ``uniform convergence on every finite set'', are the balls $$B(f;x_1,\dots,x_m;\epsilon)=\{g\in\cM(X,Y)|\sup_{1\leq i\leq m}d(f(x_i),g(x_i))<\epsilon\}.$$
\end{itemize}
Thus for a compact set $K\subset X$, we define $$B(f;K;\epsilon)=\{g\in\cM(X,Y)|\sup_{x\in K}d(f(x),g(x))<\epsilon\}.$$
\begin{lemma}[Compact Convergence Topology]
    Let $X$ be a topological space and $Y$ be a metric space. Then $$\cB_{c.c.}=\{B(f;K;\epsilon)|f\in\cM(X,Y),K\subset X\text{ is compact, }\epsilon>0\}$$
    is a topological basis of $\cM(X,Y)$. The topology $\cT_{c.c.}$, called the \textbf{compact convergence topology}, it generates satisfies:
    $$ f_n\text{ converges to }f\text{ on every compact subset}\Longleftrightarrow f_n\text{ converges to }f\text{ in }\cT_{c.c.}.$$
\end{lemma}

Back to our original problem. Let $f_n\in\mathcal{C}(X,Y)$ and $f_n\to f_0$ under $\cT_{c.c.}$,
then $f_0$ is continuous on every compact subset. But is $f_0$ continuous?
\begin{example}
    
\end{example}
To remedy, we introduce the following concept
\begin{definition}[Locally Compact Space]
    If every point in $X$ has a compact neighborhood, then $X$ is called a \textbf{locally compact space}.
\end{definition}
\begin{proposition}
    If $X$ is a locally compact space, then $\mathcal{C}(X,Y)$ is closed in $(\cM(X,Y),\cT_{c.c.})$.
\end{proposition}
In most applications, locally compact spaces are also Hausdorff. A locally compact Hausdorff space is often shorthanded by a \textbf{LCH} space.
\begin{proposition}
    Let $X$ be a LCH space, $K$ be a compact set, $U\supset K$ be an open set. Then there exists an open set $V$ such that $\overline{V}$ is compact and 
    $$ K\subset V\subset \overline{V}\subset U .$$
\end{proposition}

Although locally compactness , it is not the weakest condition that we can come up with.
\begin{definition}[Compactly Generated Space]
    
\end{definition}

When $Y$ is a general topological space (not a metric space any more), we cannot define compact convergence toplogy on $\cM(X,Y)$.
However, using standard topological arguments, we can define
\begin{definition}[Compact-Open Topology]
    
\end{definition}

Now we begin to study the compactness of 
\begin{definition}[Relatively Compact (Precompact)]
    $A\subset X$. If $\bar{A}$ is compact, we say that $A$ is \textbf{precompact(relatively compact)}.
\end{definition}

\begin{definition}
    Let $\cF\subset \mathcal{C}(X,Y)$ be a family of continuous maps.
    For $a\in X$, let $\cF_a:=\{f(a)|f\in\cF\}$.
    \begin{enumerate}[label=(\roman*)]
        \item $\cF$ is \textbf{pointwise bounded} if for any $a\in X$ $\cF_a$ is bounded in $Y$.
        \item $\cF$ is \textbf{pointwise precompact} if for any $a\in X$ $\cF_a$ is precompact in $Y$.
    \end{enumerate}
\end{definition}

\begin{theorem}[Arzela-Ascoli]
    Let $X$ be a compact space, $(Y,d)$ be a metric space, and $\cF\subset\mathcal{C}(X,Y)$ be equipped with $\cT_{c.c.}$.
    \begin{enumerate}[label=(\roman*)]
        \item If $\cF$ is equicontinuous and pointwise precompact, then $\cF$ is precompact in $(\mathcal{C}(X,Y),\cT_{c.c.})$.
        \item If in addition $X$ is a LCH, then the converse is also true.
    \end{enumerate}
\end{theorem}
\begin{proof}
    
\end{proof}
\begin{remark}
    Be careful that the conclusion stated compactness only, instead of sequential compactness.
    Thus in general we can not obtain the existence of a convergent subsequence.
\end{remark}
However, when \textit{$X$ is compact}, then $\cT_{c.c.}=\cT_{u.c.}$. As $\cT_{u.c.}$ is induced by a metric, compactness is equivalent to sequential compactness in this case.


When $X$ is only a LCH,
by the diagonal argument
\begin{definition}[$\sigma$-Compact Space]
    If a topological space $X$ can be written as a countable union of compact subsets, then $X$ is called \textbf{$\sigma$-compact}.
\end{definition}

\subsection{Application: Stone-Weierstrass Theorem}




\section{Countability Axioms}

\begin{definition}
    Let $(X,\cT)$ be a topological space
    \begin{enumerate}
        \item If each point of $X$ has a countable neighbourhood basis, then $X$ is called a \textbf{first-countable} space.
        \item If $X$ has a countable topological basis, then $X$ is called a \textbf{second-countable} space.
    \end{enumerate}
\end{definition}
\begin{remark}
    second countability $\Longrightarrow$ first countability
\end{remark}

benefit of first cou

\begin{definition}[Separability]
    Let $(X,\cT)$ be a topological space. If $X$ contains a countable dense subset, then $X$ is called a \textbf{separable} space.
\end{definition}

\begin{theorem}[Second Countability $\Longrightarrow$ Separability]
    
\end{theorem}
\begin{proof}
    Let $\{U_n\}$ be the countable topological basis. For each $n$ we can choose $x_n\in U_n$. Let $A=\{x_n\}$. $A$ is countable.
    For any $x\in X$ and an open neighborhood $U$ of $x$, there exists $n$ such that $x\in U_n\subset U$. Thus $U\cap A\neq \emptyset$ and $\overline{A}=X$ by Proposition \ref{prop:x in overline A}.
\end{proof}

\begin{proposition}
    A metric space $(X,d)$ is second countable if and only if it is separable.
\end{proposition}

\section{Separation Axioms}

TODO:This section is too abstract please give some examples

\begin{definition}
    Suppose $(X,\cT)$ is a topological space.
    \begin{enumerate}
        \item If for any $x\neq y$, there exists open sets $U,V$ such that $$x\in U\setminus V\text{ and } y\in V\setminus U,$$
        then $X$ is called a \textbf{Frechet(T1)} space.
        \item If for any $x\neq y$, there exists open sets $U,V$ such that $$x\in U,\, y\in V,\text{ and } U\cap V=\emptyset,$$
        then $X$ is called a \textbf{Hausdorff(T2)} space.
        \item If for any closed set $A$ and $x\notin A$, there exists open sets $U,V$ such that $$A\subset U,\, x\in V,\text{ and }U\cap V=\emptyset,$$
        then $X$ is called a \textbf{regular(T3)} space.
        \item If for any closed sets $A\cap B=\emptyset$, there exists open sets $U,V$ such that $$A\subset U,\,B\subset V,\text{ and }U\cap V=\emptyset,$$
        then $X$ is called a \textbf{normal(T4)} space.
    \end{enumerate}
\end{definition}

T1+T4 -> T3, T1+T3 ->T2

\subsection{Urysohn Metrization}


\section{Connectedness}

\begin{definition}
    Let $(X,\cT)$ be a topological space.
    If there exists nonempty subsets $A,B\subset X$ such that $$X=A\cup B\text{ and }A\cap\overline{B}=\overline{A}\cap B=\emptyset,$$ then $X$ is \textit{not} connected.
\end{definition}

\begin{proposition}
    TFAE:
    \begin{enumerate}
        \item $X$ is not connected.
        \item There exists disjoint open sets $A,B\subset X$ such that $X=A\cup B$.
        \item There exists disjoint closed sets $A,B\subset X$ such that $X=A\cup B$.
        \item There exists $A\neq \emptyset,A\neq X$ such that $A$ is both open and closed in $X$.
        \item There exists a continuous surjection $f:X\to\{0,1\}$. 
    \end{enumerate}
\end{proposition}
\begin{proof}
    
\end{proof}