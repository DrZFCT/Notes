\chapter{Basic Topology}




To faciliate the study of convergence on an aribitary space

We use metric space as example, for the sake of functional analysis.

\section{Basic Topology}

\subsection{Definitions of Topology}
neighborhood, open set, closed set, 

\subsubsection{Neighborhood}
Let $X$ be a (possibly empty) set. 
Let $\cN$ be a function assigning to each $x\in X$ a non-empty collection $\cN(x)$ of subsets of $X$.
The elements of $\cN(x)$ will be called neighbourhoods of $x$.

\subsubsection{Interior}


\subsubsection{Open}



\subsubsection{Closed}


\subsection{Comparing Different Topologies}

\begin{definition}
    Let $\cT_1$ and $\cT_2$ be two topologies on $X$. We say $\cT_1$ is weaker than $\cT_2$ if $\cT_1\subset \cT_2$.
\end{definition}

\subsection{Convergence and Continuity}
\begin{definition}
    Let $x_n$. If exists $x_0\in X$ such that for any neighborhood
\end{definition}

\begin{definition}
    Let $f:(X,\cT_X)\to(Y,\cT_Y)$
    \begin{enumerate}[label=(\roman*)]
        \item continuity
        \item sequential continuity
    \end{enumerate}
\end{definition}
\begin{theorem}
    continuity implies sequential continuity
\end{theorem}


\begin{definition}[Open and Closed Mappings]
    Let $f:(X,\cT_X)\to(Y,\cT_Y)$
    \begin{enumerate}[label=(\roman*)]
        \item open mapping
        \item closed mapping
    \end{enumerate}
\end{definition}


\subsection{Topological Basis}



\begin{equation*}
    \cT_\cB=\{U\subset X|\forall x\in U,\exists B\in\cB \text{ such that } x\in B\subset U\}
\end{equation*}

for which $\cB$, $\cT_\cB$ defined as above is a topology.
\begin{equation*}
    \forall x\in X,\exists B\in\cB\text{ such that }x\in B.
\end{equation*}
\begin{equation*}
    \forall B_1,B_2\in\cB,\forall x\in B_1\cap B_2,\exists B\in\cB\text{ such that }x\in B\subset B_1\cap B_2
\end{equation*}


\begin{theorem}
    \begin{equation*}
        \cT_\cB=\{\cup_{B\in\cB'}B|\cB'\subset \cB\}.
    \end{equation*}
\end{theorem}


\subsection{Constructing New }

\subsubsection{Subspace Topology}

\subsubsection{Product Topology}

\begin{definition}
    Let $(Y_\alpha,\cT_\alpha)$ be a family of,
    let 
    $$
    \cF={f_\alpha:X\to (Y_\alpha,\cT_\alpha)}
    $$
    be a family of mappings.
    The the weakest topology on $X$ such that any $f_\alpha$ is continuous
\end{definition}

\begin{example}[Weak and Weak* Topology]
    Let $X$ be a topological vector space and $X^*$ be its dual,
    $$X^*=\{f:X\to\KK|f\text{ linear and continuous}\}.$$
    \begin{enumerate}[label=(\roman*)]
        \item weak topology: $X^*$-induced topology
        \item weak* topology: $\{\ev_x|x\in X\}$-induced topology
    \end{enumerate}
\end{example}


\subsection{Sets in Topological Space}





\begin{definition}[Limit and Sequential Limit]
    \begin{enumerate}[label=(\roman*)]
        \item limit point (accumulation point)
        \item sequential limit
    \end{enumerate}
\end{definition}

A closed set has sequential limit.
\begin{theorem}
    Let $A\subset X$ be closed. If $x_n\in A$ and $x_n\to x\in X$, then $x\in A$.
\end{theorem}

a closed set also has its limit point, more precisely,
\begin{theorem}
    $A\subset X$ is closed if and only if $A'\subset A$.
\end{theorem}

closure $\bar{A}:=A\cup A'$
\begin{theorem}
    For any $A\subset X$, $A\cup A'$ is closed.
\end{theorem}
\begin{corollary}
    \begin{equation*}
        A\cup A'=\bigcup_{A\subset F \text{ closed }}F.
    \end{equation*}
\end{corollary}

interior is the dual of closure 

\begin{definition}[Dense and Nowhere Dense]
    
\end{definition}

\begin{definition}
    
\end{definition}


\section{Compactness}



\begin{definition}[Compactness and Sequential Compactness]
    \begin{enumerate}[label=(\roman*)]
        \item Every open covering has finite sub covering.
        \item Every sequence has a convergent subsequence.
    \end{enumerate}
    
\end{definition}
Leveraging the duality of open and closed set, we can immediately


What about subsets? The canonical definition is the following
\begin{remark}
    Let $A\subset X$ be a subset. If $(A,\cT_{subspace})$ is compact, then we say $A$ is compact in $X$.
\end{remark}
We can see that the above definition reduce to the common 
\begin{proposition}
    Let $A\subset X$ be a subset. $A$ is compact in $X$ if and only if 
\end{proposition}
For sequential compactness, the definition is straightforward.


\begin{theorem}
    Let $f:X\to Y$ be a continuous map.
    \begin{enumerate}[label=(\roman*)]
        \item If $A\subset X$ is compact, then $f(A)\subset Y$ is compact.
        \item If $A\subset X$ is sequentially compact, then $f(A)\subset Y$ is sequentially compact.
    \end{enumerate}
\end{theorem}
\begin{corollary}
    quotient space
\end{corollary}

\begin{theorem}
    Let $A\subset X$ be closed.
    \begin{enumerate}[label=(\roman*)]
        \item If $X$ is compact, then $A$ is compact.
        \item If $X$ is sequentially compact, then $A$ is sequentially compact.
    \end{enumerate}
\end{theorem}


\subsection{Product of Compact}

\begin{theorem}[Tychonoff]
    If $X_\alpha$ is compact for all $\alpha$, then $(\prod_{\alpha}X_{\alpha},\cT_{product})$ is also compact.
\end{theorem}

\subsection{Example: Metric Spaces}

\begin{definition}[Total Boundedness]
    Let $(X,d)$ be a metric space. If for any $\epsilon>0$, there exists a finite number of $\epsilon$ balls that cover $X$.
\end{definition}

\begin{definition}[$\epsilon$ net]
    
\end{definition}

\begin{proposition}
    total boundedness is equivalent to finite $\epsilon$ net
\end{proposition}


\begin{definition}[Absolutely Closed]
    
\end{definition}

\begin{theorem}
    In a metric space $(X,d)$, TFAE:
    \begin{enumerate}[label=(\roman*)]
        \item $A$ is compact
        \item $A$ is sequentially compact
        \item $A$ is \textbf{totally} bounded and complete.
    \end{enumerate}
\end{theorem}
\begin{remark}
    Generalization of compactness on $\RR$.
\end{remark}



\section{Topologies on Mappings}
should move this to chapter 1.

For any set $X$ and topological space $Y$, we can consider 
$$
\cM(X,Y):=\{f:X\to Y\}.
$$
We would like to study the topologies on $\cM(X,Y)$ that leverage the mapping structure.

\begin{enumerate}[label=(\roman*)]
    \item product topology = pointwise convergence topology $\cT_{p.c.}$
    \item uniform convergence topology $\cT_{u.c.}$
\end{enumerate}

Suppose $(X,\cT)$ is a topological space and $Y$ is a metric space, then we can focus on continuous maps.
$$
\mathcal{C}(X,Y):=\{f\in \cM(X,Y)|f\text{ continuous}\}
$$

compact convergence $\cT_{c.c.}$

when $Y$ is not a metric space
compact open

\subsection{Compactness in ...}

\begin{definition}[Relatively Compact (Precompact)]
    $A\subset X$. If $\bar{A}$ is compact, we say that $A$ is relatively compact.
\end{definition}

\begin{definition}
    Let $\cF\subset \mathcal{C}(X,Y)$ be a family of continuous map
    For $a\in X$, let $\cF_a:=\{f(a)|f\in\cF\}$.
    \begin{enumerate}[label=(\roman*)]
        \item pointwise bounded if for any $a\in X$ $\cF_a$ is bounded in $Y$.
        \item 
    \end{enumerate}
\end{definition}

\begin{theorem}[Arzela-Ascoli]
    
\end{theorem}

\section{Countability}

\begin{definition}
    \begin{enumerate}
        \item C1
        \item C2
    \end{enumerate}
\end{definition}



\section{Separability}

\begin{definition}
    \begin{enumerate}
        \item T1 Frechet
        \item T2 Hausdorff
        \item T3
        \item T4
    \end{enumerate}
\end{definition}