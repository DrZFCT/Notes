\chapter{Banach Spaces I}

We can regard linear functional analysis as an extension of linear algebra to infinite-dimensional cases,
with $\mathbb{F}=\mathbb{R}$, or $\mathbb{F}=\mathbb{C}$.
In classical linear algebra we utilize the powerful concept of dimension to derive the structure of linear spaces and linear transformations.
Here to cope with infinity, we must resort to techniques from mathematical analysis.





\section{Basic Definitions and Examples}
\begin{definition}[Seminorm]
    If $V$ is a vector space over $\mathbb{F}$, a seminorm is a function $p:V\to [0,\infty)$ with:\newline 
    (i) $p(x+y)\le p(x)+p(y)$, $\forall x,y\in V$ \newline
    (ii) $p(\alpha x)=\left|\alpha\right|p(x)$, $\forall \alpha\in \mathbb{F}$, $\forall x\in V$.
\end{definition}
Note that (ii) implies $p(0)=0$.
\begin{definition}[Norm]
    A norm is a seminorm $p$ s.t. $x=0$ if $p(x)=0$. Usually a norm is denoted by $\|\cdot\|$.
    A normed space is a vector space endowed with with a norm. A Banach space is a normed space that is complete w.r.t. the metric defined by the norm.
\end{definition}

\begin{lemma}[Continuity]
    If $V$ is a normed space, then: \newline 
    (i) $V\times V\to V$, $(x,y)\mapsto x+y$ is continuous. \newline
    (ii) $\mathbb{F}\times V\to V$, $(\alpha,x)\mapsto \alpha x$ is continuous.
\end{lemma}
\begin{proof}
    By the definiton of continuity and the triangle inequality of norm.
\end{proof}

\begin{lemma}
    If $p$ and $q$ are seminorms on $V$, TFAE: \newline
    (i) $p\le q$ \newline 
    (ii) $p<1$ whenever $q<1$\newline 
    (iii) $p\le 1$ whenever $q\le 1$ \newline 
    (iv) $p\le 1$ whenever $q<1$.
\end{lemma}
\begin{proof}
    Only need to show (iv) implies (i). \newline 
    Suppose (iv). Fix any $x$, and set $\alpha=q(x)$. Then for any $\epsilon>0$, $q(\frac{1}{\alpha+\epsilon}x)<1$.
    So $p(\frac{1}{\alpha+\epsilon}x)\le 1$ and $p(x)\le q(x)+\epsilon$. The arbitariness of $\epsilon$ implies $p\le q$.
\end{proof}

\begin{definition}[Equivalent Norms]
    If $\|\cdot\|_1$ and $\|\cdot\|_2$ are two norms on $V$, they are said to be equivalent norms if they define
    the same topology on $V$.
\end{definition}

\begin{lemma}
    
\end{lemma}

\section{Linear Operators on Normed Spaces}
$B(V,W)=$ all continuous linear transformations from $V$ to $W$.


\begin{lemma}
    If $V$ and $W$ are normed spaces and $T:V\to W$ is a linear transformation, TFAE:\newline 
    (i) $T\in B(V,W)$. \newline 
    (ii) $T$ is continuous at $0$. \newline 
    (iii) $T$ is continuous at some point.\newline 
    (iv) $\exists c>0$ s.t. $\forall x\in V$ $\|Tx\|\le c\|x\|$.
\end{lemma}

\section{Finite Dimensional Normed Spaces}
\begin{theorem}
    If $V$ is a finite dimensional vector space, then any two norms on $V$ are equivalent.
\end{theorem}
\begin{proof}
    Let $\{ e_1,\cdots,e_d\}$ be a Hamel basis for $V$. For $x=\sum_{j=1}^{d}x_je_j$, define $\|x\|_\infty:=\max\{|x_j|:1\le j\le d\}$.
    Then $\|x\|_\infty$ is a norm. Let $\|\cdot\|$ be any norm on $V$.If $x=\sum_{j=1}^{d}x_je_j$, then $\|x\|\le C\|x\|_\infty$, 
    where $C=\sum_{j=1}^{d}\|e_j\|$. To show the other inequality, we need a technique from analysis (or topology).

    ???
\end{proof}
\begin{corollary}
    A finite dimensional linear manifold $M$ in a normed space $V$ is closed.
\end{corollary}
\begin{proof}
    Choose a Hamel basis for $M$ and define a norm $\|\cdot\|_\infty$ as above. Then $M$ is complete w.r.t $\|\cdot\|_\infty$,
    thus complete w.r.t the original norm by the above theorem. Hence it is closed.
\end{proof}
\begin{corollary}
    A linear transformation $T$ from a finite dimensional normed space $V$ to any normed space $W$ is continuous.
\end{corollary}
\begin{proof}
    Since all norms are equivalent on $V$, we may choose a Hamel basis and define $\|\cdot\|_\infty$ as above.
    Thus for $x=\sum_{j=1}^{d}x_je_j$, $\|Tx\|=\|\sum_j x_j Te_j\|\le \sum_j |\xi_j| \|Te_j\|\le C\|x\|_\infty$, where $C=\sum_j \|Te_j\|$.
    Hence $T$ is bounded and continuous.
\end{proof}

\section{Quotients and Products of Normed Spaces}
Let $V$ be a normed space, let $M$ be a linear manifold in $V$, and let $Q:V\to V/M$ be the natural map $Qx=x+M$.
Our goal is to make $V/M$ into a normed space, so define \[\|x+M\|=\inf \{\|x+y\|:y\in M\}=\text{dist}(x,M).\]
This defines a seminorm on $V/M$, but if $M$ is not closed, it can not define a norm.
\begin{theorem}
    If $M\le V$ and $\|\cdot\|$ is defined above, then it is a norm on $V/M$. Also:\newline 
    (i)
\end{theorem}

\section{The Hahn-Banach Theorem}
We first state the Hahn-Banach Theorem for real spaces and then extend it to complex spaces. Thereafter,
we discuss several corollary of it. We postpone the proof to the end of this section.
\begin{definition}[Sublinear Functional]
    If $V$ is a vector space over $\mathbb{R}$, a seminorm is a function $q:V\to \mathbb{R}$ with:\newline 
    (i) $q(x+y)\le q(x)+q(y)$, $\forall x,y\in V$ \newline
    (ii) $q(\alpha x)=\alpha q(x)$, $\forall \alpha\ge 0$, $\forall x\in V$.
\end{definition}

\begin{theorem}[Hahn-Banach]
    Let $V$ be a vector space over $\mathbb{R}$ and let $q$ be a sublinear functional on $V$. If $M$ is a
    linear manifold in $V$ and $f:M\to\mathbb{R}$ is a linear functional s.t. $\forall x\in M$, $f(x)\le q(x)$,
    then there is a linear functional $F:V\to\mathbb{R}$ s.t. $F|M=f$ and $\forall x\in V$, $F(x)\le q(x)$.
\end{theorem}
Note that the essence of the theorem is not the extension exists but that an extension can be found that remains
dominated by $q$. Just to find an extension, we can simply take a Hamel basis.

\begin{lemma}[Complexification]
    Let $V$ be a linear space over $\mathbb{C}$.\newline 
    (i) If $f:V\to\mathbb{R}$ is an $\mathbb{R}$-linear functional, then $\tilde{f}(x)=f(x)-if(ix)$ is a
    $\mathbb{C}$-linear functional and $f=\Re \tilde{f}$. \newline 
    (ii) If $g:V\to\mathbb{C}$ is $\mathbb{C}$-linear, $f=\Re g$, and $\tilde{f}$ is defined as in (i), then 
    $\tilde{f}=g$. \newline
    (iii) If $p$ is a seminorm on $V$, and $f$ and $\tilde{f}$ are as in $(i)$, then $\forall x$ $\left|f(x)\right|\le p(x)$
    if and only if $\forall x$ $|\tilde{f}(x)|\le p(x)$.\newline 
    (iv) If $V$ is furthermore a normed space, and $f$ and $\tilde{f}$ are as in $(i)$, then 
    $\left\|f\right\|=\|\tilde{f}\|$.
\end{lemma}


\begin{theorem}[Seperation of Convex Sets]
    
\end{theorem}

\section{The Opening Mapping Theorem}
\begin{theorem}[Open Mapping Theorem]
    If $V$ and $W$ are Banach spaces and $T:V\to W$ is a continuous linear surjection,
    then $T(G)$ is open in $W$ whenever $G$ is open in $V$.
\end{theorem}


\begin{theorem}[Inverse Mapping Theorem]
    If $V$ and $W$ are Banach spaces and $T:V\to W$ is a bounded linear transformation that is bijective,
    then $T^{-1}$ is bounded.
\end{theorem}
\begin{proof}
    As $T$ is continuous and surjective, by the open mapping theorem $T$ is open and hence a homeomorphism.
\end{proof}

\begin{definition}[isomorphism between Banach spaces]
    If $V$ and $W$ are Banach spaces, an isomorphism of $V$ and $W$ is a linear bijection that is a homeomorphism.
    Say $V$ and $W$ are isomorphic if there is an isomorphism of $V$ onto $W$.
\end{definition}
\begin{remark}
    The use of the word 'isomorphism' is counter to the spirit of category theory, but it is traditional in Banach 
    space theory. The inverse mapping theorem just says that a continuous bijection is an isomorphism.
\end{remark}


\begin{theorem}[Closed Graph Theorem]
    If $V$ and $W$ are Banach spaces and $T:V\to W$ is a linear transformation s.t. the graph of $T$,
    \[ G=\left\{x\oplus Tx\in V\oplus_1 W: x\in V\right\}\] 
    is closed, then $T$ is continuous.
\end{theorem}
\begin{proof}
    Since $V\oplus_1 W$ is a Banach space and $G$ is closed, $G$ is a Banach space.
    Define $P_1:G\to V$ by $P_1(x\oplus Tx)=x$ and define $P_2:G\to W$ by $P_2(x\oplus Tx)=Tx$. 
    $P_1$ and $P_2$ are bounded. Moreover, $P_1$ is bijective. By the inverse mapping theorem, $P_1^{-1}$ is continuous. 
    Thus $T=P_2\circ P_1^{-1}$ is continuous.
\end{proof}


\section{The Principle of Uniform Boundness}
\begin{theorem}
    Let $X$ be a Banach space and $Y$ a normed space. If $\mathcal{T}\subset B(X,Y)$ s.t. $\forall x\in V$,
    $\sup\left\{\left\|Tx\right\|:T\in \mathcal{T}\right\}<\infty$, then $\sup\left\{\left\|T\right\|:T\in\mathcal{T}\right\}<\infty$.
\end{theorem}



\section{weak and weak*}

weaker topologies, easier to converge 

\begin{definition}[weakly bounded]
    $E\subset X$ is a weakly bounded set if 
    for any $f\in X^*$, there exists $M_f>0$ such that 
    $$ |f(x)|\leq M_f,\quad \forall x\in E. $$
\end{definition}

\begin{theorem}
    weak bounded if and only if bounded (in norm)
\end{theorem}

\begin{theorem}
    $x_n\overset{w}{\to} x$ if and only if 
    \begin{enumerate}[label=(\roman*)]
        \item $\{\|x_n\|\}$ is bounded;
        \item $\exists G\subset X^*$, $\span G$ is dense in $X^*$ and $\forall f\in G$, $$f(x_n)\to f(x).$$
    \end{enumerate}
\end{theorem}

weakly sequential compact

\begin{theorem}
    Every convex subset $E$ of a Banach space $X$ that is closed (in the strong topology) is closed in the weak topology.
\end{theorem}

\section{The Adjoint of a Linear Operator}


\section{The Banach-Stone Theorem}



\section{Compact Operators}
\begin{definition}
    If $V$ and $W$ are Banach spaces and $T:V\to W$ is a linear transformation,
    then $T$ is compact if ??? is compact in $W$.
\end{definition}
