\documentclass{article}
\usepackage[utf8]{inputenc}
\usepackage[a4paper,total={6in,10in}]{geometry}
\usepackage{amsmath}
\usepackage[all,cmtip]{xy}

\usepackage{amsthm}

\newtheorem{theorem}{Theorem}[section]
\newtheorem{corollary}{Corollary}[theorem]
\newtheorem{lemma}{Lemma}[section]
\newtheorem{example}{Example}[section]
\newtheorem*{remark}{Remark}

\theoremstyle{definition}
\newtheorem{definition}{Definition}[section]


\usepackage{amsfonts}
\usepackage{amssymb}
\usepackage{hyperref}



\title{Counterfactuals}
\author{Kaizhao Liu}
\date{May 2023}

\begin{document}
\maketitle
\tableofcontents

This note aims to present the basic idea of counterfactuals in the language of probability theory. 
In this way, I hope to introduce new concepts and structures to enrich probability theory, and borrow tools and insights from probability theory to study causal inference.
In the following I fix $(\Omega,\mathcal{F},\mathbb{P})$ to be a probability space.
\section{Binary Treatment}
 Let $X,Y$ be random variables on $\Omega$.
\begin{example}[binary treatment]\label{binary treatment}
    You can imagine $\Omega$ as the collection of all people being investigated, each element $\omega\in\Omega$ stands for a single person being investigated.
    Suppose $X$ is a binary treatment variable, where $X=1$ means 'treated' and $X=0$ means 'not treated'. Let $Y$ be some outcome varible such as the absence of disease.
    The goal is to study the relationship between $Y$ and $X$. 
\end{example}

\begin{definition}[potential decomposition]
    If a random variable $Y$ can be written as $Y=C_01_{A^c}+C_11_A$ where $C_0$ and $C_1$ are two random variables,
    then we say that $Y$ admits a potential decomposition $C_0,C_1$ w.r.t. $A$.
\end{definition}
\begin{remark}
    If $X=1_A$ is the binary treatment varible associated with $A$, then we also say that $Y$ admits a potential decomposition w.r.t $X$. 
    In this case, we can call $C_0,C_1$ the potential outcomes with the following interpretation: $C_0$ is the outcome if not 
    treated and $C_1$ is the outcome if treated.
\end{remark}

\begin{theorem}[existence]
    For any random variable $Y$ on $\Omega$ and any event $A\in\mathcal{F}$, there exists random variable $C_0$ and $C_1$ s.t. 
    \[Y=C_01_{A^c}+C_11_{A}.\]
\end{theorem}
This theorem is self-evident. We can look at the following cases, where we assume $X=1_A$ is a binary treatment variable. 
\begin{example}
    Let $C_0=C_1=Y$, then it is a potential decomposition of $Y$ w.r.t. $X$. In this example, the outcome is the same whether treated or not.
    We can interpret this as $X$ has no causal effect on $Y$.
\end{example}

\begin{example}
    Let $C_0=Y1_{D}$ and $C_1=Y1_E$, where $A^c\subset D$, $A\subset E$ and $D,E\in\mathcal{F}$, then it is a potential decomposition of $Y$ w.r.t. $X$. 
    In this example, $D,E$ can be chosen rather arbitarily. This shows that potential decomposition is not unique.
\end{example}
The problem of the above example is that we can decompose a random variable \textit{a posteriori}. 
To model the causal effect of the real world, we want the decomposition to be \textit{a priori}.
Namely, we are given a treatment $X$ and potential outcomes $C_0,C_1$ first, then we construct $Y$ naturally. 
We express this special type of potential decomposition more succinctly by
\begin{equation}
    Y=C_X,
\end{equation}
which is called the \textbf{consistency relationship}.


Now we can define statistics.

\begin{definition}[average causal effect]
    Define the average causal effect or average treatment effect to be
    \begin{equation}
        \theta=\mathbb{E}C_1-\mathbb{E}C_0.
    \end{equation}
\end{definition}

\section{}
\begin{definition}[counterfactual function]
    A random variable which is parameterized by $X$.
\end{definition}

\begin{definition}[causal regression function]
    Define the causal regression function to be 
    \begin{equation}
        \theta(x)=\mathbb{E}_\omega C(x,\omega).
    \end{equation}
    Note that $x$ is fixed.
\end{definition}

\end{document}