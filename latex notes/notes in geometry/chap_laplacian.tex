\chapter{The Laplacian}

\section{Divergence and Laplacian}
Let $M^n$ be a manifold with a metric $g$ and associated Levi-Civita connection $\nabla$.

derivative of $u\in C(M)$:
\begin{itemize}
    \item gradient $\nabla u\in\Gamma^{1,0}(M)$
    \item differential $du\in\Gamma^{0,1}(M)$.
\end{itemize}
they are dual via the metric 
\[\left\langle\nabla u,V\right\rangle=du(V)=V(u).\]

\begin{definition}[Hessian]
    The Hessian is a (0,2) tensor defined by $\nabla du$, which satisfies 
    \begin{equation}
        \hess_u(V,W)=\left\langle \nabla_V\nabla u,W\right\rangle.
    \end{equation}
\end{definition}

\begin{proposition}
    $\hess_u$ is symmetric.
\end{proposition}
We will give two proofs. 
\begin{proof}[Proof 1]
    First,
    \[\hess_u(\partial_i,\partial_j)=u_{ij}-\sum_{k}\Gamma^k_{ij}u_k\]
\end{proof}
\begin{proof}[Proof 2]
    \begin{align*}
        \hess_u(V,W)-\hess_u(W,V)&=\left\langle \nabla_V\nabla u,W\right\rangle-\left\langle \nabla_W\nabla u,V\right\rangle\\
        &=V\left\langle \nabla u,W\right\rangle-\left\langle  \nabla u,\nabla_V W\right\rangle -W\left\langle \nabla u,V\right\rangle+\left\langle \nabla u,\nabla_W V\right\rangle\\
        &=V(W(u))-W(V(u))-\left\langle  \nabla u,[V,W]\right\rangle=0.
    \end{align*}
\end{proof}

\begin{definition}[Divergence]
    The divergence $\div V$ of a vector field $V$ is the trace of the $(1,1)$ tensor $\nabla V$.
\end{definition}

In an orthonormal frame 
\[\div V=\sum_i\left\langle \nabla_{e_i}V,e_i\right\rangle\]

In a local coordinate,
\[\div V=\partial_i V^i+\Gamma^{j}_{ij}V^i\]
But we also have 
\begin{proposition}\label{prop:divergence}
    \begin{equation}
        \div V=\frac{1}{\sqrt{\det g}}\sum_i \partial_i(\sqrt{\det g}V^i).
    \end{equation}
\end{proposition}
\begin{proof}
    This is shown by 
    \[\frac{\partial_i \sqrt{\det g}}{\sqrt{\det g}}=\sum_j \Gamma^j_{ij}\]
\end{proof}

\Cref{prop:divergence} makes it possible to extend the Euclidean divergence theorem to manifolds. Recall the volume element $dv_g=\sqrt{\det g}\d{x}$, we have 
\begin{equation}
    \div(V)dv_g=\div_{\RR^n}(V\sqrt{\det g})\d{x}.
\end{equation}

From Leibniz rule 
\begin{equation}
    \div (uV)=u\div(V)+\left\langle \nabla u,V\right\rangle
\end{equation}

\begin{definition}[Laplacian]
    The Laplacian is the divergence of $\nabla u$.
\end{definition}

Therefore, we have in an orthonormal frame
\begin{align*}
    \Delta u &=\sum_i\left\langle \nabla_{e_i}\nabla u,e_i\right\rangle\\
    &=\sum_i\nabla_{e_i}\left\langle \nabla u,e_i\right\rangle - \left\langle \nabla u,\nabla_{e_i}e_i\right\rangle\\
    &=\sum_i\nabla_{e_i}\nabla_{e_i} u -  \nabla_{\nabla_{e_i}e_i} u 
\end{align*}
which is in accordance with \Cref{eq:connection-applied-twice}

\subsection{Bochner Formula}
\begin{proposition}
    If $u\in C(M)$, then 
    \begin{equation}
        \Delta\nabla u=\nabla (\Delta u)+\ric(\nabla u,\cdot)
    \end{equation}
\end{proposition}

\begin{proof}[Proof 1]
    
\end{proof}

geodesic normal coordinates for $p\in M$ fixed 

\begin{itemize}
    \item $g_{ij}(p)=\delta_{ij}$, meaning that $\partial_i$ are orthonormal at $p$
    \item $\Gamma^k_{ij}=0$, meaning that $\nabla \partial_i(p)=0$
\end{itemize}


\begin{proof}[Proof 2]
    
\end{proof}

\begin{corollary}\label{cor:bochner}
    \begin{equation}
        \Delta|\nabla u|^2=|\hess_u|^2+\left\langle \nabla (\Delta u),\nabla u\right\rangle+\ric(\nabla u,\nabla u)
    \end{equation}
\end{corollary}

\subsection{Lichnerowicz Theorem}


\section{Submanifold Divergence and Laplacian}

In this section, $\Sigma\subset M$ is a submanifold with the induced connection $\bar{\nabla}$.

\begin{proposition}
    The submanifold Hessian is given by
    \begin{equation}
        \overline{\hess}_u(V,W)=\hess_u(V,W)+\left\langle A(V,W),\nabla^\perp u\right\rangle
    \end{equation}
\end{proposition}
\begin{proof}
    \begin{align*}
        \overline{\hess}_u(V,W)&=\left\langle \nabla_V\nabla^T u,W\right\rangle\\
        &=\left\langle \nabla_V\nabla u,W\right\rangle - \left\langle \nabla_V\nabla^\perp u,W\right\rangle\\
        &=\hess_u(V,W)+  \left\langle \nabla^\perp u,\nabla_V W\right\rangle\\
        &=\hess_u(V,W)+\left\langle A(V,W),\nabla^\perp u\right\rangle
    \end{align*}
\end{proof}

\section{Laplacian Comparison}

\subsection{Distance Function}

\begin{definition}[Cut Point]
    
\end{definition}

\begin{theorem}
    If $q\in \cut(p)$, then either 
    \begin{enumerate}
        \item $q$ is the first conjugate to $p$ along a minimizing geodesic, or
        \item $q$ is the first point along a minimizing geodesic where there is a second, different, minimizing geodesic from $p$.
    \end{enumerate}
\end{theorem}

Now we show that $d$ is smooth away from $\cut(p)$.
\begin{proposition}
    
\end{proposition}


$\triangle |x|^2$ on $\RR^n$

$\nabla|x|^2=2|x|\nabla|x|$

$\triangle |x|=\frac{n-1}{|x|}$ on $\RR^n\setminus\{0\}$

Another way 
$\hess_{|x|}$

Suppose $\ric\geq 0$

\subsection{Laplacian Comparison}

We first state the Laplacian comparison for smooth points
\begin{theorem}
Suppose that $\ric \geq 0$ and $r(x) = d(p, x)$ for $p$ fixed. Away from $\cut(p) \cup \{p\}$, we have that
\[
\Delta r \leq \frac{n-1}{r}.
\]
\end{theorem}
\begin{proof}
    First, $\hess_r$ has rank at most $n-1$, so 
    \[|\hess_r|^2\geq \frac{(\Delta r)^2}{n-1}.\]
    By Bochner formula (\Cref{cor:bochner}), along $\gamma$, 
    \begin{align*}
        0&=\frac{1}{2}\Delta|\nabla r|^2 \\
        &=|\hess_r|^2+\left\langle \nabla (\Delta r),\nabla r\right\rangle+\ric(\nabla r,\nabla r)\\
        &\geq \frac{(\Delta r)^2}{n-1}+ (\Delta r)'
    \end{align*}
\end{proof}

\begin{lemma}
    If $f(t)$ is a function on $(0, T]$ with $(n - 1) f' \leq -f^2$, then
\[
f(T) \leq \frac{n - 1}{T}.
\]
\end{lemma}
\begin{proof}
    Notice that
    \[\left(\frac{1}{f(t)}\right)'=-\frac{f'(t)}{f(t)^2}\]
\end{proof}

\begin{definition}[barrier]
We say that $\Delta f \geq g$ at $p \in \Omega$ in the \emph{barrier sense} if for every $\epsilon > 0$
there exists a $C^2$ function $h_\epsilon$ and an open set $U_\epsilon$ containing $p$ so that
\begin{enumerate}
    \item $f(p) = h_\epsilon(p)$ and $h_\epsilon \leq f$ in $U_\epsilon$.
    \item $\Delta h_\epsilon(p) \geq g(p) - \epsilon$.
\end{enumerate}
\end{definition}

\begin{definition}[viscosity]
We say that $\Delta f \geq g$ at $p \in \Omega$ in the \emph{viscosity sense} if for every open
set $U$ containing $p$ and $C^2$ function $\varphi$ on $U$ with $f(p) = \varphi(p)$ and $\varphi \geq f$ in $U$, we have that
$\Delta \varphi(p) \geq g(p)$.
\end{definition}

\begin{proposition}
If $\Delta f \geq g$ at $p$ in the barrier sense, then it also holds in the viscosity sense. 
If $\Delta f \geq g$ at $p$ in the viscosity sense, then it also holds in the distributional sense. 
\end{proposition}

\begin{theorem}
    If $\ric\geq 0$ and $d(x):=d(p,x)$ for some fixed $p$, then $\Delta d\leq \frac{n-1}{d}$ in the barrier sense on $M\setminus\{p\}$.
\end{theorem}

\subsection{Bishop-Gromov Volume Comparison}

\begin{theorem}
    If $\ric\geq 0$, $p\in M^n$, and $0<r_1<r_2$, then 
    \[\frac{\vol(B_{r_2}(p))}{r_2^n}\leq \frac{\vol(B_{r_1}(p))}{r_1^n}.\]
\end{theorem}

\subsection{Dirichlet Poincare Inequality}
Given a compact domain $\Omega\subset M$, a \textit{Dirichlet Poincare inequality} is an inequality of the form 
\begin{equation}\label{eq:Dirichlet-Poincare}
    \int_\Omega u^2\leq C_\Omega\int_\Omega |\nabla u|^2,
\end{equation}
where $u$ is required to vanish on $\partial \Omega$.
\begin{theorem}
    If $M^n$ has $\ric\geq 0$ and $\Omega=B_R(p)$, then \eqref{eq:Dirichlet-Poincare} holds with $C_\Omega=3^{2n+4}R^2$.
\end{theorem}
\begin{proof}
    We apply the divergence theorem to $u^2\nabla w$, where $w$ is to be chosen.
    \begin{align*}
        0&=\int\div(u^2\nabla w)\\
        &=\int u^2\Delta w+2u\left\langle \nabla u,\nabla w\right\rangle \\
        &\geq \int u^2 \Delta w -2|u||\nabla u||\nabla w|
    \end{align*}
    If we can find a function $w$ such that $\Delta w$ is lower bounded and $|\nabla w|$ is upper bounded by some constants depending only on $\Omega$,
    then 
\end{proof}

\section{Gradient Estimates}

\begin{theorem}
    If $B_R(p) \subset M^n$ has $\ric \geq 0$, $\Delta u = 0$, and $u > 0$, then
\[
\sup_{B_{R/2}(p)} |\nabla \log u| \leq \frac{C}{R},
\]
where $C$ depends just on $n$.
\end{theorem}
differential Harnack Inequality
\begin{corollary}
    
\end{corollary}

Bernstein technique
\begin{lemma}
    If $\Delta u=0$, $u>0$, and $\ric\geq 0$, then $w=\log u$ satisfies 
    \begin{itemize}
        \item $\Delta w=-|\nabla w|^2 $
        \item $\frac{1}{2}\Delta|\nabla w|^2\geq \frac{1}{n}|\nabla w|^4-\left\langle\nabla w,\nabla|\nabla w|^2\right\rangle$
    \end{itemize}
\end{lemma}
\begin{proof}
    
\end{proof}

But $|\nabla w\|^2$ may not have an interior max.
Use cut-off 

\begin{proof}
    
\end{proof}

Meanvalue Inequality

for harmonic is evident due to Harnack, extend it to sub-harmonic
\begin{theorem}
    If $M^n$ has $\ric\geq 0$ and $v\geq 0$ and $\Delta v\geq 0$ on $B_{4R}(p)$, then 
    \begin{equation}
        \sup_{B_R(p)}v^2\leq C_n \frac{\int_{B_{4R}(p)}v^2}{\vol(B_{4R}(p))}
    \end{equation}
\end{theorem}
\begin{proof}
    $\phi$ cutoff $1$ on $B_{2R}(p)$,
    reverse Poincare 
    \[\int_{B_{2R}(p)}|\nabla v|^2\leq 4\int_{B_{4R}(p)}v^2|\nabla\phi|^2.\]
    Choose $\phi$ such that $|\nabla\phi|\leq\frac{1}{2R}$,
    \[\int_{B_{2R}(p)}|\nabla v|^2\leq \frac{1}{R^2}\int_{B_{4R}(p)}v^2.\]
    \[\div(\phi^2v\nabla v)=\phi^2|\nabla v|^2+\phi^2v\Delta v+\]
    solve for $u$ on $B_{2R}(p)$ such that $\Delta u=0$ in $B_{2R}(p)$ and $u=v$ on $\partial B_{2R}(p)$.
    \[v\leq u.\]
    use Harnack inequality on $u$
\end{proof}


\section{}


\begin{theorem}[Colding-Minicozzi]
    $\ric\geq 0$, $\dim\cH^d(M^n)\leq Cd^{n-1}$
\end{theorem}

\begin{theorem}
    $\exists C=C(n)$ such that if $u_1,\dots,u_N$ are $L^2(B_{2r})$ orthonormal and $\Delta u_i=0$,
    and $\int_{B_r}u_i^2\geq \alpha>0$, then $N\leq\frac{C}{\alpha}$.
\end{theorem}

\begin{lemma}
    Given $x\in B_{2r}$, $\exists y\in S^{N-1}$ such that $w=\sum_{i=1}^{N}y_i u_i$ has $\sum u_i^2(x)=w^2(x)$
\end{lemma}
\begin{proof}
    Define $f:S^{n-1}\to\RR$, $f(y)=\sum_{i=1}^{N}y_i u_i(x)$,
    max achieved when.
\end{proof}

\begin{proof}
    given $x\in B_r$, $\sum_{i=1}^{N}u_i^2(x)=w^2(x)$ where $w$ is from the lemma.
    \[w^2(x)\leq \frac{C}{\vol(B_r(x))}\int_{B_r(x)}w^2\leq \frac{C}{\vol(B_r(x))}\]
    Bishop-Gromov: 
\end{proof}

\begin{theorem}
    If $v_1,\dots,v_{2N}\in\cH^d(M^n)$ and are linearly independent.
    Then $\exists R>0$ and $u_1,\dots,u_N$ in the span of the $v_i$'s such that 
    $\int_{B_{2R}}u_iu_j=\delta_{ij}$ and $\int_{B_R}u_i^2>2^{-4(d+n)}$
\end{theorem}

\begin{lemma}
    $0<F\leq Cr^d$ on $[1,\infty)$, then $\exists \infty$ many $k\in\NN$ such that 
    \[\frac{F(2^{k+1})}{F(2^k)}\leq 2^{d+\epsilon}.\]
\end{lemma}
\begin{proof}
    by contradiction
\end{proof}

\begin{proof}
    Define \[\Lambda_j=\{v_1,\dots,v_{j-1}\subset \cH^d\}.\]
    Given $r$, define $w_{j,r}$ to be the $L^2(B_r)$ projection of $v_j$ onto $\Lambda_j$.
    Define $f_j(r)=\int_{B_r}(v_j-w_{j,r})^2\leq \int_{B_r}(v_j-w)^2$
\end{proof}
