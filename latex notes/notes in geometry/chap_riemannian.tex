\chapter{Riemannian Manifold}


\section{Riemannian Metrics}



\section{Affine Connections}

An affine connection $\nabla_{\cdot}(\cdot)$ is a map from $\Gamma(M)\times \Gamma(M)$ to $\Gamma(M)$ 

\begin{definition}
    \begin{enumerate}
        \item $C(M)$-linear in the lower slot 
        \item $\RR$-linear in the upper slot
        \item $C(M)$-Leibniz rule in the upper slot 
    \end{enumerate}
\end{definition}

\subsection{Levi-Civita Connection}
Let $\nabla$ be an affine connection, then by the basis theorem there are locally defined functions $\Gamma^k_{ij}$ such that 
\begin{equation}
    \nabla_{\partial_i}\partial_j=\sum_k \Gamma^k_{ij}\partial_k.
\end{equation}

A connection is symmetric if 



metric compaibility

\begin{theorem}
    Given a metric $g$, there is a unique connection $\nabla$ that is symmetric and metric compatible.
    This connection is called the \textbf{Levi-Civita connection}, and is given by
\end{theorem}

\subsection{Tensor Leibniz Rule}


\begin{equation*}
    \nabla_{E_i}E_j=\Gamma^k_{ij}E_k
\end{equation*}


\begin{proposition}
    \begin{equation*}
        F^{i_1\dots i_k}_{j_1\dots j_l;m}=E_m(F^{i_1\dots i_k}_{j_1\dots j_l})+\sum_{s=1}^{k}\Gamma^{i_s}_{mp} F^{i_1 \dots p \dots i_k}_{j_1\dots j_l}-\sum_{s=1}^{l}\Gamma^{p}_{mj_s} F^{i_1 \dots i_k}_{j_1\dots p \dots j_l}
    \end{equation*}
\end{proposition}

higher order covariant derivative can be computed by iterating


\subsection{Along a Curve}

\section{Curvature}

The covariant derivative of a $(r,s)$ tensor can be thought of as an $(r,s+1)$ tensor in a natural way.
For example, if 

Repeating this, we get a $(1,2)$ tensor $\nabla\nabla V$, which will abbreviate as $\nabla^2 V$.

Using the Leibniz rule, we see that 
\begin{equation*}
    \nabla_X(\nabla_Y V)=\nabla^2_{X,Y}V+\nabla_{\nabla_X Y}V
\end{equation*}

The tensor is not necessarily symmetric in the two lower slots. In fact, the curvature comes in 

\begin{align*}
    \nabla^2_{X,Y}V-\nabla^2_{Y,X}V&= \nabla_X(\nabla_Y V)-\nabla_Y(\nabla_X V)-\nabla_{\nabla_X Y-\nabla_Y X}V\\
    &=\nabla_X(\nabla_Y V)-\nabla_Y(\nabla_X V)-\nabla_{[X,Y]}V\\
    &=R(Y,X)V
\end{align*}
This is known as the \textbf{Ricci identity}.

\begin{definition}[Riemann curvature]
    
\end{definition}

\subsection{Symmetries}
