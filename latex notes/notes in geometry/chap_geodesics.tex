\chapter{Geodesics and Minimal Submanifolds}

\section{Geodesics Equation and Exponential Map}
\begin{equation}
    \nabla_{\gamma'}\gamma'=0
\end{equation}

The exponential map $\exp_p:T_pM\to M$
the differential $(\d \exp_p)_v:T_pM\to T_{\exp_p(v)}M$
\begin{proposition}
    \[(d\exp_p)_0(v)=v\]
\end{proposition}

\subsection{Gauss Lemma}

Let $v,w\in T_p M$ be vectors with $\left\langle v,w\right\rangle=0$ and define a map $F:\RR^2\to M$ by
\begin{equation*}
    F(s,t)=\exp_p(t(v+sw)).
\end{equation*}
Define vector fields $F_s$ and $F_t$ by 
\[F_s=d{F}(\partial_s)\]

\begin{lemma}[Gauss Lemma]
    \[|F_t(s,t)|^2=|v|^2+s^2|w|^2\]
    \[\left\langle F_s,F_t\right\rangle(0,t)=0.\]
\end{lemma}
\begin{remark}
    It is insightful to rewrite the conclusions of the Gauss Lemma as
    \[\left\langle (d\exp_p)_v(v),(d\exp_p)_v(w)\right\rangle=\left\langle v,w\right\rangle\]
    for $v,w\in T_pM$.
\end{remark}

\subsection{Hopf-Rinow Theorem}
The Riemannian distance $d(p,q)$ between points $p,q\in M$ is defined to be the infimum over piece-wise smooth curves $\gamma$ from $p$ to $q$ of the length of $\gamma$.
\begin{remark}
    We choose piece-wise smooth curves as definition because it is convenient to work with.
    In the end, we will see that...
\end{remark}

\section{Variational Theory of Geodesics}
Given a Riemannian manifold $(M,g)$ and a curve $\gamma:[0,a]\to M$, a variation of $\gamma$ is a mapping $F:[-\epsilon,\epsilon]\times [0,a]\subset \RR^2$ so that 


Jacobi equation 

\begin{definition}[Conjugate Points]
    Suppose $\gamma$ is a geodesic. We say that $\gamma(t_2)$ is conjugate to $\gamma(t_1)$ along $\gamma$ if there is a non-zero Jacobi field $J$ along $\gamma$ so that 
    $J(t_1)=J(t_2)=0$ 
\end{definition}

The energy of a piece-wise smooth curve $\gamma:[0,a]\to M$
\begin{equation}
    \bE(\gamma)=\int_{0}^{a}|\gamma'|^2\d{t}.
\end{equation}

The first 
\begin{proposition}
    \[\bE'(0)=-2\int_0^a\left\langle F_s,\nabla_{F_t}F_t\right\rangle\d{t}-\]
\end{proposition}


Define the index form $I(V,V)$ by 
\[I(V,V)=\int_0^a |\nabla_{\gamma'}V|^2-\R(\gamma',V,\gamma',V)\]
\begin{proposition}
    The second variation of energy at $0$ is 
    \[\frac{1}{2}\bE''(0)=I(V,V)\]
\end{proposition}

\begin{theorem}[Bonnet-Myers]\label{thm:Bonnet-Myers}
    If $(M^n,g)$ is complete with $\ric\geq c>0$, then $M$ is compact and 
    \[\diam^2(M)\leq (n-1)\frac{\pi^2}{c}\]
\end{theorem}
\begin{proof}
    We bound the length of any stable geodesic. Thus, suppose $\gamma:[0,a]\to M$ is a stable geodesic.
    Let $e_1,\dots,e_{n-1}$ be a parallel orthonormal frame along $\gamma$ and define variation vector fields $V_1,\dots,V_{n-1}$ by 
    \[V_j=\left(\sin\frac{\pi t}{a}\right)e_j.\]
    Note that 
\end{proof}

\section{Variational Theory of Minimal Submanifolds}

Given an isometrically embedded submanifold $\Sigma\subset M$, a variation is a map 
\[F:\Sigma\times (-\epsilon,\epsilon)\to M\]
so that $F(x,0)=x$. The variation vector field $F_s=\d{F}(\partial_s)$ describes the motion of points in $\Sigma$ under the variation.

\subsection{First Variation}

\begin{proposition}
    \begin{equation}
        \partial_s \overline{dv}=(\div(F_s^T)-\left\langle F_s^\perp,\bH\right\rangle)\overline{dv}
    \end{equation}
\end{proposition}

\subsection{Monotonicity}


\subsection{Second Variation}
We now investigate the second derivative of volume for a hypersurface 
\[\Sigma^n\subset M^{n+1},\]
whose calculations are substantially simpler than a general submanifold.

\begin{lemma}
    Given a normal variation $F_s=u\bn$, the derivative of $\bH$ at 0 is 
    \begin{equation}
        \bH'=(\Delta u+|A|^2u+\ric(\bn,\bn))\bn-\left\langle \bH,\bn\right\rangle\nabla u.
    \end{equation}
\end{lemma}