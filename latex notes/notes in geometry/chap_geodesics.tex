\chapter{Geodesics and Minimal Submanifolds}



\section{Variational Theory of Geodesics}

Jacobi equation 

\begin{definition}[Conjugate Points]
    Suppose $\gamma$ is a geodesic. We say that $\gamma(t_2)$ is conjugate to $\gamma(t_1)$ along $\gamma$ if there is a non-zero Jacobi field $J$ along $\gamma$ so that 
    $J(t_1)=J(t_2)=0$ 
\end{definition}

The energy of a piece-wise smooth curve $\gamma:[0,a]\to M$
\begin{equation}
    \bE(\gamma)=\int_{0}^{a}|\gamma'|^2\d{t}.
\end{equation}


\section{Variational Theory of Minimal Submanifolds}

Given an isometrically embedded submanifold $\Sigma\subset M$, a variation is a map 
\[F:\Sigma\times (-\epsilon,\epsilon)\to M\]
so that $F(x,0)=x$. The variation vector field $F_s=\d{F}(\partial_s)$ describes the motion of points in $\Sigma$ under the variation.

\subsection{First Variation}

\begin{proposition}
    \begin{equation}
        \partial_s \overline{dv}=(\div(F_s^T)-\left\langle F_s^\perp,\bH\right\rangle)\overline{dv}
    \end{equation}
\end{proposition}

\subsection{Monotonicity}


\subsection{Second Variation}
We now investigate the second derivative of volume for a hypersurface 
\[\Sigma^n\subset M^{n+1},\]
whose calculations are substantially simpler than a general submanifold.

\begin{lemma}
    Given a normal variation $F_s=u\bn$, the derivative of $\bH$ at 0 is 
    \begin{equation}
        \bH'=(\Delta u+|A|^2u+\ric(\bn,\bn))\bn-\left\langle \bH,\bn\right\rangle\nabla u.
    \end{equation}
\end{lemma}