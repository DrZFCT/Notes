
\chapter{The Probabilistic Method}
\section{The Method}



\section{123 Theorem}
As a sort of 'inverse' of the probabilistic method, combinatorial techniques can also apply to probabilistic statement.


\section{The Local Lemma}

\begin{definition}[dependency digraph]
    Let $A_i$($i\in [n])$ be $n$ events. A directed graph $D=(V,E)$ on the set of vertices 
    $V=\{1,\cdots,n\}$ is called a dependency digraph for the events if 
    for each $i$ the event is mutually independent of all the events $\{A_j:(i,j)\notin E\}$.
\end{definition}
\begin{theorem}[the local lemma]
    Let $A_i$($i\in [n])$ be $n$ events. Suppose $D=(V,E)$ is a dependency digraph for the above events and 
    suppose there are real numbers $x_1,\cdots,x_n$ s.t. $1\le x_i<0$ and $P(A_i)\le x_i\prod_{(i,j)\in E}(1-x_j)$
    for all $1\le i\le n$. Then 
    \[ P(\bigcap_{i=1}^n\overline{A_i})\geq \prod_{i=1}^n (1-x_i)\] 
\end{theorem}
\begin{remark}
    In particular, with positive probability, no event $A_i$ holds.
\end{remark}

\begin{corollary}[the local lemma: symmetric case]
    
\end{corollary}
\begin{remark}
    The constant $e$ is the best possible constant.
\end{remark}


\section{Correlation Inequalities}
\begin{theorem}[the four functions theorem]
    
\end{theorem}



\begin{theorem}[FKG inequality]
    Let $L$ be a finite distributive lattice, and let $\mu:L\to \mathbb{R}^+$ be a log-supermodular function.
    Then for any two increasing functions $f,g$, we have
    \[E(fg)\geq (Ef) (Eg)\] where the expectation is taken w.r.t. 
\end{theorem}
\begin{remark}
    If both $f$ and $g$ are decreasing, the the result still holds. If one is increasing and one is decreasing, then the inequality is reversed.
\end{remark}

\begin{lemma}[Kleitman's lemma]
    Let 
\end{lemma}

\begin{theorem}
    
\end{theorem}

\begin{definition}[linear extension]
    
\end{definition}

\begin{theorem}[XYZ theorem]
    Let 
\end{theorem}

