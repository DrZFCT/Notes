\chapter{Markov Semigroup Theory}

The core idea of Markov semigroup theory is to encode the behavior of a Markov process $(X_t)_{t\geq 0}$ via operators which act on functions.
We can then develop calculus rules for working with these operators, and study them using tools from functional analysis.
This is analogous to how the linear algebraic study of the transition matrix of a discrete-time Markov chain reveals properties (e.g., ergodicity, convergence) of the chain.

\section{Diffusions}



\subsection{Kolmogorov's Theory???}


\subsection{Ito's theory???}


\section{Markov Semigroup Theory}

\begin{definition}[generator]
    \begin{equation*}
        \mathscr{L}f=\lim_{t\searrow 0}\frac{P_t f-f}{t},
    \end{equation*}
\end{definition}


\begin{definition}[carre du champ]
    \begin{equation*}
        \Gamma(f,g):=\frac{1}{2}(\mathscr{L}(fg)-f\mathscr{L}g-g\mathscr{L}f).
    \end{equation*}
\end{definition}

\begin{definition}[Dirichlet energy]
    \begin{equation*}
        \mathscr{E}(f,g):=\int \Gamma(f,g)\d \pi
    \end{equation*}

    \begin{equation*}
        \mathscr{E}(f,g)=-\int f\mathscr{L}g\d \pi
    \end{equation*}
\end{definition}

Carre du champ is nonnegative.

\begin{proposition}
    
\end{proposition}

\begin{definition}[iterated carre du champ]
    \begin{equation*}
        \Gamma_2(f,g):=\frac{1}{2}(\mathscr{L}\Gamma(f,g)-\Gamma(f,\mathscr{L}g)-\Gamma(g,\mathscr{L}f)).
    \end{equation*}
\end{definition}

\begin{definition}[Bakry-Emery ]
    \begin{equation*}
        \Gamma_2(f,f)\geq \alpha \Gamma(f,f)
    \end{equation*}
    curvature-dimension condition $\textnormal{CD}(\alpha,\infty)$.
\end{definition}

The key point is that once the (iterated) carre du champ operators are known, the curvature-dimension condition amounts to an algebraic condition which can be easily checked.


\begin{example}[Langevin diffusion]
    \begin{equation}\label{eq:langevin diffusion}
        \d X_t=-\nabla V(X_t)\d t+\sqrt{2}\d B_t
    \end{equation}

    \begin{equation*}
        \mathscr{L}f(x)=\Delta f(x)-\left\langle \nabla V(x),\nabla f(x)\right\rangle.
    \end{equation*}
\end{example}

\section{Functional Inequalities}

\subsection{Poincare Inequality}

\begin{definition}[Poincare inequality]
    \begin{equation*}
        \var_{\pi}(f)\leq \frac{C_\textnormal{PI}}{2}\mathscr{E} (f,f )
    \end{equation*}
\end{definition}


\subsection{Log-Sobolev Inequality}

\begin{definition}[log-Sobolev inequality]
    \citep{bakry}
    \begin{equation*}
        \Ent_\pi(f^2)\leq 2C_\textnormal{LSI}\mathscr{E}(f,f)
    \end{equation*}

    \cite{chewi}
    \begin{equation*}
        \KL(\rho\pi \mmid \pi)\leq \frac{C_\textnormal{LSI}}{2}\mathscr{E} (\rho,\log\rho )
    \end{equation*}

\end{definition}

\begin{example}
    For the Langevin diffusion, the LSI reads
    
    \begin{equation*}
        \KL(\rho\pi \mmid \pi)\leq \frac{C_\textnormal{LSI}}{2}\mathscr{E} (\rho,\log\rho )
    \end{equation*}


    \begin{equation*}
        \Ent_\pi(f^2)\leq 2C_\textnormal{LSI}\EE_\pi \|\nabla f\|^2  
    \end{equation*}
\end{example}


\begin{definition}[modified log-Sobolev inequality]
    \citep{vanhandel}
    \begin{equation*}
        \Ent_\pi(f)\leq 2C_\textnormal{MLSI}\mathscr{E}(f,\log f)
    \end{equation*}
\end{definition}
It seems to be equivalent to lsi if $\mathscr{L}$ works normally, for example in the case of diffusion.

\section{Transport Inequalities}

\begin{definition}[$T_2$-inequality]
    \begin{equation}
        W_2^2(\mu,\pi)\leq 2C_{\textnormal{T}_2} \KL(\mu\mmid \pi)
    \end{equation}
\end{definition}

\begin{definition}[$T_1$-inequality]
    \begin{equation}
        W_1^2(\mu,\pi)\leq 2C_{\textnormal{T}_1} \KL(\mu\mmid \pi)
    \end{equation}
\end{definition}

$T_2$ implies PI

the curvature-dimension condition is equivalent to a gradient bound

\begin{definition}[diffusion semigroup]
    
\end{definition}