\chapter{Markov Semigroup Theory}

The core idea of Markov semigroup theory is to encode the behavior of a Markov process $(X_t)_{t\geq 0}$ via operators which act on functions.
We can then develop calculus rules for working with these operators, and study them using tools from functional analysis.
This is analogous to how the linear algebraic study of the transition matrix of a discrete-time Markov chain reveals properties (e.g., ergodicity, convergence) of the chain.

\section{Diffusions}



\subsection{Kolmogorov's Theory???}


\subsection{Ito's theory???}


\section{Markov Semigroup Theory}

\begin{definition}[generator]
    \begin{equation*}
        \mathscr{L}f=\lim_{t\searrow 0}\frac{P_t f-f}{t},
    \end{equation*}
\end{definition}


\begin{definition}[carr\'e du champ]
    \begin{equation*}
        \Gamma(f,g):=\frac{1}{2}(\mathscr{L}(fg)-f\mathscr{L}g-g\mathscr{L}f).
    \end{equation*}
\end{definition}

\begin{definition}[Dirichlet energy]
    \begin{equation*}
        \mathscr{E}(f,g):=\int \Gamma(f,g)\d \pi
    \end{equation*}

    \begin{equation*}
        \mathscr{E}(f,g)=-\int f\mathscr{L}g\d \pi
    \end{equation*}
\end{definition}

carr\'e du champ is nonnegative.

\begin{proposition}
    
\end{proposition}

\begin{definition}[iterated carr\'e du champ]
    \begin{equation*}
        \Gamma_2(f,g):=\frac{1}{2}(\mathscr{L}\Gamma(f,g)-\Gamma(f,\mathscr{L}g)-\Gamma(g,\mathscr{L}f)).
    \end{equation*}
\end{definition}

\begin{definition}[Bakry-Emery ]
    \begin{equation*}
        \Gamma_2(f,f)\geq \alpha \Gamma(f,f)
    \end{equation*}
    curvature-dimension condition $\textnormal{CD}(\alpha,\infty)$.
\end{definition}

The key point is that once the (iterated) carr\'e du champ operators are known, the curvature-dimension condition amounts to an algebraic condition which can be easily checked.


\begin{example}[Langevin diffusion]
    \begin{equation}\label{eq:langevin diffusion}
        \d X_t=-\nabla V(X_t)\d t+\sqrt{2}\d B_t
    \end{equation}

    \begin{equation*}
        \mathscr{L}f(x)=\Delta f(x)-\left\langle \nabla V(x),\nabla f(x)\right\rangle.
    \end{equation*}
\end{example}

\begin{definition}[diffusion semigroup]
    The Markov semigroup $(P_t)_{t\geq 0}$ is a diffusion semigroup if for all functions $f,g\in L^2(\pi)$ in the domain of the carr\'e du champ $\Gamma$ and all $\phi:\RR\to\RR$,
    the chain rule holds:
    \begin{equation}\label{eq:diffusion_semigroup_chain_rule}
        \Gamma(\phi\circ f,g)=\phi'(g)\Gamma(f,g).
    \end{equation}
\end{definition}

For the chain rule satisfied by, which is reminiscent of Ito's lemma.
\begin{proposition}
    Equivalently, the chain rule can be stated for the generator as 
    \begin{equation*}
        \mathscr{L}\phi(f)=\phi'(f)\mathscr{L}f+\phi''(f)\Gamma(f,f).
    \end{equation*}
\end{proposition}
\begin{proof}
    
\end{proof}

\section{Functional Inequalities}

\subsection{Poincare Inequality}

\begin{definition}[Poincare inequality]
    \begin{equation*}\tag{$\mathsf{PI}$}
        \var_{\pi}(f)\leq \frac{C_\mathsf{PI}}{2}\mathscr{E} (f,f )
    \end{equation*}
\end{definition}


\subsection{Log-Sobolev Inequality}

\begin{definition}[log-Sobolev inequality]
    \citep{bakry}
    \begin{equation*}\tag{$\mathsf{LSI}$}
        \Ent_\pi(f^2)\leq 2C_\mathsf{LSI}\mathscr{E}(f,f)
    \end{equation*}

    \cite{chewi}
    \begin{equation*}
        \KL(\rho\pi \mmid \pi)\leq \frac{C_\mathsf{LSI}}{2}\mathscr{E} (\rho,\log\rho )
    \end{equation*}

\end{definition}

\begin{example}
    For the Langevin diffusion, the LSI reads
    
    \begin{equation*}
        \KL(\rho\pi \mmid \pi)\leq \frac{C_\textnormal{LSI}}{2}\mathscr{E} (\rho,\log\rho )
    \end{equation*}


    \begin{equation*}
        \Ent_\pi(f^2)\leq 2C_\textnormal{LSI}\EE_\pi \|\nabla f\|^2  
    \end{equation*}
\end{example}


\begin{definition}[modified log-Sobolev inequality]
    \citep{vanhandel}
    \begin{equation*}
        \Ent_\pi(f)\leq 2C_\textnormal{MLSI}\mathscr{E}(f,\log f)
    \end{equation*}
\end{definition}
It seems to be equivalent to lsi if $\mathscr{L}$ works normally, for example in the case of diffusion.

The entropy has an additional $\log$ factor, thus we can see that log Sobolev inequality.
Another way to see this is through a simple linearization argument.
\begin{lemma}[$\mathsf{LSI} \Rightarrow \mathsf{PI}$]
    The log-Sobolev inequality implies the Poincare inequality.
\end{lemma}
\begin{proof}
    Suppose $\int f\d \pi=0$, the RHS of $\mathsf{LSI}$ can be linearized by
    $$\KL((1+\epsilon f)\pi\mmid \pi)=\frac{\epsilon^2}{2}\int f^2\d\pi +o(\epsilon^2)$$
\end{proof}

\begin{theorem}
    TFAE:
    \begin{enumerate}
        \item log-Sobolev
        \item For any function $f$ and all $t\geq 0$,
        \begin{equation*}
            \Ent_\pi (P_tf)\leq \exp\left(-\frac{2t}{C_\mathsf{LSI}}\right)\Ent_\pi(f)
        \end{equation*}
        \item For any initial distribution $\pi_0$ and all $t\geq 0$,
        \begin{equation*}
            \KL(\pi_t\mmid \pi)\leq \exp\left(-\frac{2t}{C_\mathsf{LSI}}\right) \KL(\pi_0\mmid \pi)
        \end{equation*}
    \end{enumerate}
\end{theorem}

\subsection{Transport Inequalities}

\begin{definition}[$T_1$-inequality]
    \begin{equation}\label{eq:T1}\tag{$\mathsf{T_1}$}
        W_1^2(\mu,\pi)\leq 2C_\mathsf{T_1} \KL(\mu\mmid \pi)
    \end{equation}
\end{definition}

\begin{definition}[$T_2$-inequality]
    \begin{equation}\label{eq:T2}\tag{$\mathsf{T_2}$}
        W_2^2(\mu,\pi)\leq 2C_\mathsf{T_2} \KL(\mu\mmid \pi)
    \end{equation}
\end{definition}



$\mathsf{LSI} \Rightarrow \mathsf{T_2}\Rightarrow \mathsf{PI}$ 

it is evident that $\mathsf{T_2}\Rightarrow \mathsf{T_1}$. However, in general $\mathsf{T_1}$ and $\mathsf{PI}$ are incomparable.



the curvature-dimension condition is equivalent to a gradient bound


operations preserving 
\subsection{Bounded Perturbation}


