\documentclass{article}
\usepackage[utf8]{inputenc}
\usepackage[a4paper,total={6in,10in}]{geometry}
\usepackage{amsmath}
\usepackage[all,cmtip]{xy}

\usepackage{amsthm}

\newtheorem{theorem}{Theorem}[section]
\newtheorem{corollary}{Corollary}[theorem]
\newtheorem{lemma}{Lemma}[section]
\newtheorem{example}{Example}[section]
\newtheorem*{remark}{Remark}

\theoremstyle{definition}
\newtheorem{definition}{Definition}[section]


\usepackage{amsfonts}
\usepackage{amssymb}
\usepackage{hyperref}



\title{von Mises Expansion}
\author{Kaizhao Liu}
\date{July 2023}

\begin{document}
\maketitle
\tableofcontents

\section{Int}

\begin{definition}[statistical functional]
    Let $X_1,\cdots,X_n$ be a sample from population with distribution function $F$ and let $T_n=T_n(X_1,\cdots,X_n)$ be a 
    statistic. If $T_n$ can be written as a functional $T$ of the empirical distribution $\hat{F}_n$, i.e. $T_n=T(\hat{F}_n)$ where $T$
    does not depend on $n$.
\end{definition}

\begin{example}
    Let $\phi$ be a real-valued function and let 
    \[ T_n=\frac{1}{n}\sum_{i=1}^{n}\phi(X_i).\] 
    Then for a general distribution $G$, the functional defined by 
    \[ T(G)=\int \phi(x) \mathrm{d}G(x) \] 
    satisfies $T_n(X_1,\cdots,X_n)=T(\hat{F}_n)$. Functional of this form is called linear statistical functional.
\end{example}

\begin{example}
    The sample $k$-th central moment $T_n=m_k=T_n(\hat{F}_n)$, where 
    \[ T(G)=\int [x-\int x\mathrm{d}G]^k \mathrm{d}G(x)\]
\end{example}

\section{Influence Function}
\begin{definition}[von Mises derivative]
    Let $X_1,\cdots,X_n$ be a sample and let $T$ be a functional on a convex set of d.f.'s containing all empirical d.f.'s and the population d.f. $F$.
    Let $G$ be a point in this convex set. Then the von Mises derivative $T'_F$ of $T$ at $F$ is defined by 
    \[ T'_F(G-F)=\frac{\mathrm{d}}{\mathrm{d}t} T(F+t(G-F))|_{t=0},\]
    if there exists a real-valued function $\phi_F(x)$ (independent of $G$) s.t. 
    \[ T'_F(G-F)=\int \phi_F(x)\mathrm{d}(G-F)(x).\] 
    We shall normalize $\phi_F(x)$ by making 
    \[\int \phi_F(x)\mathrm{d}F (x)=0 ,\]
    and call it the influence function.
\end{definition}





\section{Infinitesimal Jackknife}
Jackknife is a general tool for estimating the variance and reducing the bias.
The basic idea of Jackknife is to drop one observations and recompute the estimate using the remaining observations.
This is done for each observation, and the results are then combined to produce an estimate.


If we assign each observation a weight, then dropping one observation is identical to giving it weight of zero.
The Infinitesimal Jackknife




\section{Perturbing a Training Input}
How would the model's prediction change if a training input were modified?



\end{document}