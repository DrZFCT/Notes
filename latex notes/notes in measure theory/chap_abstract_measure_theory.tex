\chapter{Abstract Measure Theory}


\section{}
Why $\sigma$?


\begin{definition}[generator of $\sigma$-algebra]
    \[\sigma(\cA)=\{A\subset E:A\in\cE \quad \forall \cE \supset \]
\end{definition}
\begin{remark}
    Borel $\sigma$-algebra
\end{remark}

\begin{definition}[$\pi$-system]
    $\cA$ is a collection of subsets of $E$. Then $\cA$ is called a $\pi$-system if 
    \begin{enumerate}
        \item $\emptyset\in\cA$;
        \item $A,B\in\cA$, then $A\cap B\in\cA$.
    \end{enumerate}
\end{definition}

\begin{definition}[$d$-system]
    $\cA$ is a collection of subsets 
    \begin{enumerate}
        \item $E\in\cA$;
        \item $A,B\in\cA$, $A\subset B$, then $B\setminus A \in\cA$;
        \item If $A_n\subset \cA$ such that $A_n\subset A_{n+1}$, then $\cup_{n}A_n\in\cA$.
    \end{enumerate}
\end{definition}

\begin{proposition}
    $\cA$ is a $\sigma$-algebra if and only if it is a $\pi$-system and a $d$-system.
\end{proposition}

\begin{lemma}[Dynkin's $\pi$-system lemma]
    Let $\cA$ be a $\pi$-system. Then any $d$-system that contains $\cA$ also contains $\sigma(\cA)$.
\end{lemma}
Usage
\begin{proof}
    Let $\cD$ be the intersection of all $d$-system containing $\cA$.
    We now prove that $\cD$ is a $\sigma$-algebra. As $\cD$ is already a $d$-system, we only need to prove that $\cD$ is a $\pi$-system.

    (i) If $B\in\cD$ and $A\in\cA$, then $B\cap A\subset\cD$.
    Let $\cD'=\{B\in\cD:B\cap A\in \cD\quad \forall A\in\cA\}$. $D'\supset \cA$. We check that $\cD'$ is a d-system.

    Thus $\cD'=\cD$.

    (ii) If $A,B\in\cD$, then $B\cap A\in\cD$.
    Let $\cD''=\{B\in\cD:B\cap A\in \cD\quad \forall A\in\cD\}$.
\end{proof}





\begin{definition}[Set function]
    $\cA$ be a collection of subsets of $E$ with $\emptyset\in\cA$.
    A set function $\mu:\cA\to [0,\infty]$ is a function such that $\mu(\emptyset)=0$.
\end{definition}

\begin{definition}[Increasing Set function]
    $A\subset B$, we have $\mu(A)\leq \mu(B)$.
\end{definition}