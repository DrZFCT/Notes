\chapter{Abstract Measure Theory}


\section{}
In this section, we are working on a set $E$.

\begin{definition}[$\sigma$-algebra]
    
\end{definition}
\begin{remark}
    Greek letters $\sigma$ and $\delta$ are often used when countable unions and countable intersections are involved. 
    For example, topologists call $F_\sigma$ every countable union of closed sets in a topological space ($F$ standing possibly for the French word fermé, closed) and $G_\delta$
 every countable intersection of open sets ($G$ standing for the German word Gebiet, domain, connected open set). 
 The letters $\sigma$
 and $\delta$
 are often given as Greek abbreviations of German words: $\sigma$
 as S in Summe for sum (in the sense of sum of sets, that is, union) and $\delta$
 as D in Durchschnitt for intersection, both countable. 
 Thus, in the context of measure theory, the letter $\sigma$ refers to the stability of a collection of subsets by countable union.
\end{remark}

\begin{definition}[generator of $\sigma$-algebra]
    \[\sigma(\cA)=\{A\subset E:A\in\cE \quad \forall \cE \supset \}\]
\end{definition}

\begin{example}
    Borel $\sigma$-algebra
\end{example}

\begin{definition}[$\pi$-system]
    $\cA$ is a collection of subsets of $E$. Then $\cA$ is called a $\pi$-system if 
    \begin{enumerate}
        \item $\emptyset\in\cA$;
        \item $A,B\in\cA$, then $A\cap B\in\cA$.
    \end{enumerate}
\end{definition}

\begin{definition}[$d$-system]
    $\cA$ is a collection of subsets 
    \begin{enumerate}
        \item $E\in\cA$;
        \item $A,B\in\cA$, $A\subset B$, then $B\setminus A \in\cA$;
        \item If $A_n\subset \cA$ such that $A_n\subset A_{n+1}$, then $\cup_{n}A_n\in\cA$.
    \end{enumerate}
\end{definition}
\begin{remark}
    $d$-system is also referred as $\lambda$-system.
\end{remark}

\begin{proposition}\label{prop:sigma dpi}
    $\cA$ is a $\sigma$-algebra if and only if it is a $\pi$-system and a $d$-system.
\end{proposition}
\begin{proof}
    A $\sigma$-algebra is a $\pi$ system because

    Conversely, 
\end{proof}

\begin{lemma}[Dynkin's $\pi$-system lemma]
    Let $\cA$ be a $\pi$-system. Then any $d$-system that contains $\cA$ also contains $\sigma(\cA)$.
\end{lemma}
This lemma can reduce the problem of studying a $\sigma$-algebra to the study of a $\pi$-system.
\begin{proof}
    Let $\cD$ be the intersection of all $d$-system containing $\cA$.
    We now prove that $\cD$ is a $\sigma$-algebra. 
    As $\cD$ is already a $d$-system, by Proposition \ref{prop:sigma dpi}, we only need to prove that $\cD$ is a $\pi$-system.

    (i) First we show the following property for $\cD$: If $B\in\cD$ and $A\in\cA$, then $B\cap A\subset\cD$.
    Let $$\cD'=\{B\in\cD:B\cap A\in \cD\quad \forall A\in\cA\}.$$ 
    To show that $\cD'=\cD$, we only need to check that $\cD'$ is a $d$-system containing $\cA$.
    As $\cA$ is a $\pi$-system, $D'\supset \cA$. 
    Now, we can check
    \begin{itemize}
        \item $E\in\cD'$; this is because $\cA\subset\cD$.
        \item $B_1,B_2\in\cD'$, $B_1\subset B_2$, then $B_2\setminus B_1 \in\cD'$; this is because for any $A\in\cA$, $(B_2\setminus B_1)\cap A = (B_2\cap A)\setminus (B_1\cap A ) \in\cD$.
        \item If $B_n\subset \cD'$ such that $B_n\subset B_{n+1}$, then $\cup_{n}B_n\in\cD'$; this is because for any $A\in\cA$, $(\cup_{n}B_n)\cap A=\cup_n (B_n\cap A)\in\cD$.
    \end{itemize}

    (ii) If $A,B\in\cD$, then $B\cap A\in\cD$.
    Let $$\cD''=\{B\in\cD:B\cap A\in \cD\quad \forall A\in\cD\}.$$ One can see that $\cD''$ is a $d$-system by noting that the argument holds for any $\cA\subset\cD$.
    The fact that $\cD''$ contains $\cA$ follows from property (i). Therefore, $\cD''=\cD$ and $\cD$ is a $\pi$-system.
\end{proof}





\begin{definition}[set function]
    $\cA$ be a collection of subsets of $E$ with $\emptyset\in\cA$.
    A set function $\mu:\cA\to [0,\infty]$ is a function such that $\mu(\emptyset)=0$.
\end{definition}

\begin{definition}[increasing set function]
    $A\subset B$, we have $\mu(A)\leq \mu(B)$.
\end{definition}