\chapter{Real Analysis}


\section{}

Continuity of Translation

\begin{theorem}
    Suppose $f\in L^1(\RR^n)$. Then 
    $$\|f_h-f\|_{L^1}\to 0 \quad h\to 0 .$$
\end{theorem}
\begin{proof}
    The result is clearly true
    The proof is a simple consequence of the approximation of integrable functions by continuous functions of compact support.
\end{proof}

\section{Differentiation and Integration}
For conceptual simplicity, we study $\RR$ instead of $\RR^n$ in this section.
Let us first recall what we learned in elementary calculus.

\begin{theorem}
    Let \( f \) be a \textcolor{blue}{continuous} function on \([a,b]\),
    and $F$ be the function defined, for all $x$ in $[a,b]$, by
    $$F(x)=\int_a^x f(t)\rd t.$$
    Then $F$ is uniformly continuous on $[a,b]$ and differentiable on $(a,b)$, and $$F'(x)=f(x).$$  
\end{theorem}

\begin{theorem}[Newton-Leibniz]\label{thm:newton-leibniz}
    Let \( f \) be a \textcolor{blue}{Riemann integrable} function on \([a,b]\),
    and $F$ a \textcolor{blue}{continuous} function on $[a,b]$ which is an antiderivative of $f$ in $(a,b)$:
    $$F'(x)=f(x).$$
    Then $$\int_a^b f(x)\mathrm{d}x=F(b)-F(a).$$  
\end{theorem}


\begin{definition}[Total Variation]
    The total variation of a function $f$ defined on $[a,b]$ is the quantity
    $$ V_a^b(f)=\sup_{P\in\cP}\sum_{i=0}^{n_P-1}|f(x_{i+1})-f(x_i)|,$$
    where
    $$\cP=\{P=\{x_0,\cdots,x_{n_P}\}| P \text{ is a partition of }[a,b]\text{ satisfying }x_i\leq x_{i+1}\text{ for }0\leq i\leq n_P-1\}.$$
    If $V_a^b(f)<+\infty$, then $f$ is said to be of bounded variation on $[a,b]$.
\end{definition}

\begin{example}
    The continuous function 
    $$f(x)=\left\{\begin{matrix}
        0,  &  x=0\\
        x\sin(\frac{1}{x}),  & x\neq 0
        \end{matrix}\right.$$
    is not of bounded variation on $[0,2/\pi]$. Just consider the partition $$P=\{0,\frac{2}{2n-1},\frac{2}{2n-3},\cdots,\frac{2}{3},1\}.$$
\end{example}

\begin{theorem}[Jordan Decomposition Theorem]
    \begin{equation*}
        f(x)=g(x)-h(x)
    \end{equation*}
    where $g(x)$ and $h(x)$ 
\end{theorem}
Bounded variable functions says that it has a length


Bounded variable functions are a.e. differentiable
But it does not satisfies the desirable property: Length is equal to the integral of derivative,
which is also known as the \textbf{fundamental theorem of calculus}.
It turns out that absolute continuous functions is exactly the set of functions that satisfies this property.


\begin{example}
    
\end{example}



\begin{definition}[Absolute Continuity]
    A function $f:[a,b]\to \RR$ is absolutely continuous if for every positive number $\epsilon$,  there is a positive number $\delta$ such that 
    whenever a finite sequence of pairwise disjoint sub-intervals $(x_k,y_k)$ satisfies
    $$ \sum_{k=1}^{N}(y_k-x_k)<\delta $$
    then $$\sum_{k=1}^{N}|f(y_k)-f(x_k)|<\epsilon.$$
\end{definition}


Absolutely continuous functions are of bounded variation.
\begin{proposition}
    If $f:[a,b]\to\RR$ is absolutely continuous, then it is of bounded variation on $[a,b]$.
\end{proposition}

\begin{theorem}\label{thm:abs-cont}
    If $f$ is absolute continuous function on $[a,b]$, then 
    \begin{equation*}
        f(x)-f(a)=\int_{a}^{x}f'(t)\mathrm{d}t,\quad x\in [a,b].
    \end{equation*}
\end{theorem}

It is beneficial to compare it with the ``classical'' version (Theorem \ref{thm:newton-leibniz}).




