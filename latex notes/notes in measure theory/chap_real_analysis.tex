\chapter{Real Analysis}


\section{}


\section{Differentiation and Integration}
For conceptual simplicity, we study $\RR$ instead of $\RR^n$ in this section.
Let us first recall what we learned in elementary calculus.
\begin{theorem}[Newton-Leibniz]\label{thm:newton-leibniz}
        Let \( f \) be a \textcolor{blue}{Riemann integrable} function on \([a,b]\).  
        % Add your theorem content here...
    
\end{theorem}

\begin{definition}[Bounded Variation]
    
\end{definition}

\begin{theorem}[Jordan Decomposition Theorem]
    \begin{equation*}
        f(x)=g(x)-h(x)
    \end{equation*}
    where $g(x)$ and $h(x)$ 
\end{theorem}
Bounded variable functions says that it has a length


Bounded variable functions are a.e. differentiable
But it does not satisfies the desirable property: Length is equal to the integral of derivative,
which is also known as the \textbf{fundamental theorem of calculus}.



\begin{example}
    
\end{example}



\begin{definition}[Absolute Continuous]
    
\end{definition}


\begin{theorem}\label{thm:abs-cont}
    If $f$ is absolute continuous function on $[a,b]$, then 
    \begin{equation*}
        f(x)-f(a)=\int_{a}^{x}f'(t)\mathrm{d}t,\quad x\in [a,b].
    \end{equation*}
\end{theorem}

It is beneficial to compare it with the ``classical'' version (Theorem \ref{thm:newton-leibniz}).






\end{document}